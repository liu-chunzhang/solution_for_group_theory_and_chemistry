\documentclass[a4paper]{book}

\usepackage{afterpage}
\usepackage[hypcap=false]{caption}
\usepackage{enumitem}	% 定制enumerate标号
\usepackage{geometry}
\geometry{%
	left=2cm,%
	right=2cm,%
	top=2cm,%
	bottom=2cm,%
	bindingoffset=0cm
}
\usepackage{hyperref}
\hypersetup{
    colorlinks=true,            %链接颜色
    linkcolor=blue,             %内部链接
    filecolor=magenta,          %本地文档
    urlcolor=cyan,              %网址链接
    pdftitle={Overleaf Example},
    pdfpagemode=FullScreen,
}
\usepackage[none]{hyphenat}	% 阻止长单词分在两行
\usepackage{longtable}
\usepackage{mathrsfs}	% 提供\mathscr字体
\usepackage[version=4]{mhchem}
\usepackage{multirow}
\usepackage{subcaption}

\RequirePackage[many]{tcolorbox}
\tcbset{
    boxed title style={colback=magenta},
	breakable,
	enhanced,
	sharp corners,
	attach boxed title to top left={yshift=-\tcboxedtitleheight,  yshifttext=-.75\baselineskip},
	boxed title style={boxsep=1pt,sharp corners},
    fonttitle=\bfseries\sffamily,
}

\definecolor{skyblue}{rgb}{0.54, 0.81, 0.94}

\newtcolorbox[auto counter, number within=chapter, number format=\arabic]{exercise}[1][]{
    title={Exercise~\thetcbcounter},
    colframe=skyblue,
    colback=skyblue!12!white,
    boxed title style={colback=skyblue},
    overlay unbroken and first={
        \node[below right,font=\small,color=skyblue,text width=.8\linewidth]
        at (title.north east) {#1};
    }
}

\newtcolorbox[auto counter, number within=chapter, number format=\arabic]{solution}[1][]{
%    top=2ex,
%    boxrule=0pt,
%    leftrule=1.4pt,
    title={Solution~\thetcbcounter},
    colframe=teal!60!green,
    colback=green!12!white,
    boxed title style={colback=teal!60!green},
    overlay unbroken and first={
        \node[below right,font=\small,color=red,text width=.8\linewidth]
        at (title.north east) {#1};
    }
}

\newcommand{\AO}{{\rm AO}}
\newcommand{\Heff}{H^{\rm eff,\pi}}
\newcommand{\Hp}{H^\prime}
\newcommand{\Sp}{S^\prime}
\newcommand{\RRR}{{\rm R}^3}
\newcommand\Figref[1]{Fig \ref{#1}}
\newcommand\Tableref[1]{Table \ref{#1}}
\newcommand{\orb}[1]{{\rm #1}}
\newcommand{\orbp}{\orb{p}}

\allowdisplaybreaks

\begin{document}

	\setcounter{chapter}{3}

	\begin{exercise}
		
		Determine the point groups of the following: (a) \ce{CH2ClF}; (b) \ce{NH3}; (c) \ce{BCl3}; (d) allene; (e) 1,3,5-trichlorobenzene; (f) trans-\ce{Pt(NH3)2Cl2} (considered as square planar); (g) \ce{BFCIBr}.
		
	\end{exercise}

	\begin{solution}
	
	\end{solution}
	
	\begin{exercise}
		Determine the point groups of the following octahedral compounds: (a) \ce{CoN6}; (b) \ce{CoN5A}; (c) cis-\ce{CoN4A2}; (d) trans-\ce{CoN4A2}; (e) cis-cis-\ce{CoN3A3}; (f) trans-cis-\ce{CoN3A3}.
	\end{exercise}

	\begin{solution}

	\end{solution}

	\begin{exercise}
	Determine the point groups of the following: (a) chair form of cyclohexane (ignoring the \ce{H}'s); (b) boat form of cyclohexane (ignoring the \ce{H}'s); (c) staggered \ce{C2H6}; (d) eclipsed \ce{C2H6}; (e) between staggered and eclipsed \ce{C2H6}.
	\end{exercise}
	
	\begin{solution}
		
	\end{solution}
	
	\begin{exercise}
		Determine the point groups of the following: (a) ivy leaf; (b) iris; (c) starflsh; (d) a right circular cone; (e) twin-bladed propellor; (f) rectangular bar; (g) hexagonal bathroom tile; (h) swastika; (i) tennis ball (with seam); (j) Chinese abacus (counters all in their lowest positions); (k) ying-yang.
	\end{exercise}
	
	\begin{solution}
	
	\end{solution}
	
	\begin{exercise}
		Determine the point groups of the following: (a) a square-based pyramid; (b) a right circular cone; (c) a square lamina; (d) a square lamina with the top and bottom sides painted differently; (e) a right circular cylinder; (f) a right circular cylinder with the two ends painted differently; (g) a right circular cylinder with a stripe painted parallel to the axis.
	\end{exercise}
	
	\begin{solution}
	
	\end{solution}
	
	\begin{exercise}
		What is the point group for the tris(ethylenediamine)cobalt(III) ion?
	\end{exercise}
	
	\begin{solution}
	
	\end{solution}
	
	\begin{exercise}
		For which point groups can a molecule (a) have a dipole moment, (b) be optically active?
	\end{exercise}
	
	\begin{solution}
	
	\end{solution}

\end{document}