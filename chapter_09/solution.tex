\documentclass[a4paper]{book}

\usepackage{afterpage}
\usepackage[hypcap=false]{caption}
\usepackage{enumitem}	% 定制enumerate标号
\usepackage{geometry}
\geometry{%
	left=2cm,%
	right=2cm,%
	top=2cm,%
	bottom=2cm,%
	bindingoffset=0cm
}
\usepackage{hyperref}
\hypersetup{
    colorlinks=true,            %链接颜色
    linkcolor=blue,             %内部链接
    filecolor=magenta,          %本地文档
    urlcolor=cyan,              %网址链接
    pdftitle={Overleaf Example},
    pdfpagemode=FullScreen,
}
\usepackage[none]{hyphenat}	% 阻止长单词分在两行
\usepackage{longtable}
\usepackage{mathrsfs}	% 提供\mathscr字体
\usepackage[version=4]{mhchem}
\usepackage{multirow}
\usepackage{subcaption}
\usepackage{titlesec}

\RequirePackage[many]{tcolorbox}
\tcbset{
    boxed title style={colback=magenta},
	breakable,
	enhanced,
	sharp corners,
	attach boxed title to top left={yshift=-\tcboxedtitleheight,  yshifttext=-.75\baselineskip},
	boxed title style={boxsep=1pt,sharp corners},
    fonttitle=\bfseries\sffamily,
}

\definecolor{skyblue}{rgb}{0.54, 0.81, 0.94}

\newtcolorbox[auto counter, number within=chapter, number format=\arabic]{problem}[1][]{
    title={Problem~\thetcbcounter},
    colframe=skyblue,
    colback=skyblue!12!white,
    boxed title style={colback=skyblue},
    overlay unbroken and first={
        \node[below right,font=\small,color=skyblue,text width=.8\linewidth]
        at (title.north east) {#1};
    }
}

\newtcolorbox[auto counter, number within=chapter, number format=\arabic]{solution}[1][]{
%    top=2ex,
%    boxrule=0pt,
%    leftrule=1.4pt,
    title={Solution~\thetcbcounter},
    colframe=teal!60!green,
    colback=green!12!white,
    boxed title style={colback=teal!60!green},
    overlay unbroken and first={
        \node[below right,font=\small,color=red,text width=.8\linewidth]
        at (title.north east) {#1};
    }
}

\newtcolorbox{remark}[1][]{
    title={Remark},
    colframe=yellow!45!orange,
    colback=yellow!45!white,
    coltitle=white,
    boxed title style={colback=yellow!45!orange},
    overlay unbroken and first={
        \node[below right,font=\small,color=white,text width=.8\linewidth]
        at (title.north east) {#1};
    }
}

\newcommand{\AO}{{\rm AO}}
\newcommand{\Heff}{H^{\rm eff,\pi}}
\newcommand{\Hp}{H^\prime}
\newcommand{\Sp}{S^\prime}
\newcommand{\RRR}{{\rm R}^3}
\newcommand\Figref[1]{Fig \ref{#1}}
\newcommand\Tableref[1]{Table \ref{#1}}
\newcommand\Eqref[1]{eqn (\ref{#1})}
\newcommand{\orb}[1]{{\rm #1}}
\newcommand{\orbs}{\orb{s}}
\newcommand{\orbp}{\orb{p}}
\newcommand{\orbd}{\orb{d}}
\newcommand{\orbf}{\orb{f}}

\allowdisplaybreaks

\titleformat{\chapter}[display]
  {\bfseries\Large}
  {\filright\MakeUppercase{\chaptertitlename} \Huge\thechapter}
  {1ex}
  {\titlerule\vspace{1ex}\filleft}
  [\vspace{1ex}\titlerule]

\begin{document}

	\setcounter{chapter}{8}
	
	\chapter{Molecular Vibrations}
	
	% 9.1
	\begin{problem}
		
	For ethylene:
	\begin{enumerate}[label=(\alph*)]
	
	\item determine the point group;
	
	\item determine the number and symmetry of the vibrational normal coordinates;
	
	\item determine the spectroscopic activity of each fundamental level.
	
	\end{enumerate}				
		
	\end{problem}

	\begin{solution}
	
	\begin{enumerate}[label=(\alph*)]
	
	\item Ethylene ($\text{C}_2\text{H}_4$) possesses 3 mutually perpendicular $C_2$ axes and a horizontal mirror plane ($\sigma_{\rm h}$). Consequently, it is assigned to the $\mathscr{D}_{\rm 2h}$ point group.
	
	\item The character of $\Gamma^0$ is
	\begin{center}
	\begin{tabular}{c|cccccccc} \hline
		$R=$		& $E$ & $C_2(z)$ & $C_2(y)$ & $C_2(x)$ & $i$ & $\sigma(xy)$ & $\sigma(xz)$ & $\sigma(yz)$ \\ \hline
		$\chi^0(C_i)$ & 18 & 0 & 0 & -2 & 0 & 6 & 2 & 0 \\ \hline
	\end{tabular}
	\end{center}
	Solving the system of linear equations like in Problem 7.1, we obtain
	\begin{equation*}
		\Gamma^0 = 3 \Gamma^{A_g} \oplus 3 \Gamma^{B_{1g}} \oplus 2 \Gamma^{B_{2g}} \oplus \Gamma^{B_{3g}} \oplus \Gamma^{A_u} \oplus 2 \Gamma^{B_{1u}} \oplus 3 \Gamma^{B_{2u}} \oplus 3 \Gamma^{B_{3u}} . 
	\end{equation*}
	
	From the character table of $\mathscr{D}_{\rm 2h}$, viz., \Tableref{table:character_table_of_d2h}, we find the decomposition of the translational representation $\Gamma^{\rm t}$ and the rotational representation $\Gamma^{\rm r}$:
	\begin{align*}
		\Gamma^{\rm t} &= \Gamma^{B_{1u}} \oplus \Gamma^{B_{2u}} \oplus \Gamma^{B_{3u}} , \\
		\Gamma^{\rm r} &= \Gamma^{B_{1g}} \oplus \Gamma^{B_{2g}} \oplus \Gamma^{B_{3g}} .
	\end{align*}
	
	\begin{center}
	\captionof{table}{The character table for the $\mathscr{D}_{\rm 2h}$ point group.} \label{table:character_table_of_d2h}
	\begin{tabular}{c|cccccccccc}\hline
$\mathscr{D}_{\rm 2h}$ & $E$ & $C_2(z)$ & $C_2(y)$ & $C_2(x)$ & $i$ & $\sigma(xy)$ & $\sigma(xz)$ & $\sigma(yz)$	&	&	 \\ \hline
		$A_g$	&	1	&	1	&	1	&	1	&	1	&	1	&	1	&	1	&	&	$x^2$; $y^2$; $z^2$	\\
		$B_{1g}$	&	1	&	1	&	-1	&	-1	&	1	&	1	&	-1	&	-1	&	$R_z$	&	$xy$		\\
		$B_{2g}$	&	1	&	-1	&	1	&	-1	&	1	&	-1	&	1	&	-1	&	$R_y$	&	$xz$		\\
		$B_{3g}$	&	1	&	-1	&	-1	&	1	&	1	&	-1	&	-1	&	1	&	$R_x$	&	$yz$	\\
		$A_u$	&	1	&	1	&	1	&	1	&	-1	&	-1	&	-1	&	-1	&	&	\\
		$B_{1u}$	&	1	&	1	&	-1	&	-1	&	-1	&	-1	&	1	&	1	&	$z$	&	\\
		$B_{2u}$	&	1	&	-1	&	1	&	-1	&	-1	&	1	&	-1	&	1	&	$y$	&	\\
		$B_{3u}$	&	1	&	-1	&	-1	&	1	&	-1	&	1	&	1	&	-1	&	$x$	&	\\ \hline
	\end{tabular}
	\end{center}
	
	Hence, we obtain the decomposition of the vibrational representation $\Gamma^{\rm v}$:
	\begin{equation}
		\Gamma^{\rm v} = 3 \Gamma^{A_g} \oplus 2 \Gamma^{B_{1g}} \oplus \Gamma^{B_{2g}} \oplus \Gamma^{A_u} \oplus \Gamma^{B_{1u}} \oplus 2 \Gamma^{B_{2u}} \oplus 2 \Gamma^{B_{3u}} .
	\end{equation}
	
	Therefore, we conclude that there are in total 12 vibrational normal coordinates, and 3, 2, 1, 1, 1, 2, 2 for $\Gamma^{A_g}$, $\Gamma^{B_{1g}}$, $\Gamma^{B_{2g}}$, $\Gamma^{A_u}$, $\Gamma^{B_{1u}}$, $\Gamma^{B_{2u}}$, $\Gamma^{B_{3u}}$, respectively.
	
	\item We list all determination of the spectroscopic activity of each fundamental level as follows:
	\begin{center}
	{
	\renewcommand{\arraystretch}{1.2}
	\begin{tabular}{c|c|c} \hline
		Irreducible representations & Count & Spectroscopic activity \\ \hline
			$\Gamma^{A_g}$ 		& 3 & Raman \\
			$\Gamma^{B_{1g}}$ 	& 2 & Raman \\
			$\Gamma^{B_{2g}}$ 	& 1 & Raman \\
			$\Gamma^{A_u}$ 		& 1 & inactive \\
			$\Gamma^{B_{1u}}$ 	& 1 & infra-red	\\ 
			$\Gamma^{B_{2u}}$ 	& 2 & infra-red	\\
			$\Gamma^{B_{3u}}$ 	& 2 & infra-red \\ \hline
	\end{tabular}
	}
	\end{center}
	
	Because ethylene has a center of inversion ($i$), the Rule of Mutual Exclusion, stated in the last paragraph of section 9-11, dictates that no mode can be both infra-red and Raman active. 
	
	\end{enumerate}		
	
	\end{solution}
	
	% 9.2
	\begin{problem}

	Show on the basis of infra-red and Raman spectra that it is possible to distinguish between the two crown forms of octachlorocyclooctane, one in which the hydrogen atoms are all equatorial ($\mathscr{D}_{\rm 4d}$) and the other in which the hydrogen atoms are alternating between axial and equatorial positions ($\mathscr{C}_{\rm 4v}$).
		
	\end{problem}

	\begin{solution}	
	
	We analyse these two issues separately, and summarize these conclusions in the end.
	
	\begin{itemize}
	
	\item Firstly, we list the character tables of the $\mathscr{D}_{\rm 4d}$ point group as follows. 
	\begin{center}
	\captionof{table}{The character table for the $\mathscr{D}_{\rm 4d}$ point group.}\label{table:character_table_of_d4d}
	\begin{tabular}{c|ccccccccc} \hline
		$\mathscr{D}_{\rm 4d}$ & $E$ & $2S_8$ & $2C_4$ & $2S^3_8$ & $C_2$ & $4C^\prime_2$ & $4 \sigma_{\rm d}$ &		&	\\ \hline
		$A_1$	&	1	&	1	&	1	&	1	&	1	&	1	&	1	&	&	$x^2+y^2$; $z^2$	\\
		$A_2$	&	1	&	1	&	1	&	1	&	1	&	-1	&	-1	&	$R_z$	&\\
		$B_1$	&	1	&	-1	&	1	&	-1	&	1	&	1	&	-1	&	&	\\
		$B_2$	&	1	&	-1	&	1	&	-1	&	1	&	-1	&	1	&	$z$	&\\
		$E_1$	&	2	& $\sqrt{2}$	& 0	& $-\sqrt{2}$ & -2 & 0 	&	0	& ($x$, $y$) &\\
		$E_2$	&	2	& 	0	& 	-2	& 	0 	& 	2 	& 	0 	&	0	&	& ($x^2-y^2$, $xy$)	\\
		$E_3$	&	2	& $-\sqrt{2}$ & 0 & $\sqrt{2}$ & -2 & 0 	&	0	& ($R_x$, $R_y$) & ($xz$, $yz$)	\\ \hline
	\end{tabular}
	\end{center}
	
	In the $\mathscr{D}_{\rm 4d}$ point group, the irreducible representations of the infra-red active modes and Raman active modes are mutually exclusive:
	\begin{align}
		\Gamma^{\rm t} &= \Gamma^{B_2} \oplus \Gamma^{E_1} , \\
		\Gamma^{\rm r} &= \Gamma^{A_2} \oplus \Gamma^{E_3} , \\
		\Gamma^{\alpha} &= 2 \Gamma^{A_1} \oplus \Gamma^{E_2} \oplus \Gamma^{E_3} .
	\end{align}
	
	The character of $\Gamma^0$ is
	\begin{center}
	\begin{tabular}{c|ccccccc} \hline
		$R=$		&  $E$ & $2S_8$ & $2C_4$ & $2S^3_8$ & $C_2$ & $4C^\prime_2$ & $4 \sigma_{\rm d}$ \\ \hline
		$\chi^0(C_i)$ & 72 & 0 & 0 & 0 & 0 & -2 & 12 \\ \hline
	\end{tabular}
	\end{center}
	Solving the system of linear equations like in Problem 7.1, we obtain
	\begin{equation*}
		\Gamma^0 = 6 \Gamma^{A_1} \oplus 3 \Gamma^{A_2} \oplus 3 \Gamma^{B_1} \oplus 6 \Gamma^{B_2} \oplus 9 \Gamma^{E_1} \oplus 9 \Gamma^{E_2} \oplus 9 \Gamma^{E_3} . 
	\end{equation*}
	Hence, we obtain
	\begin{equation}
		\Gamma^{\rm v} = 6 \Gamma^{A_1} \oplus 2 \Gamma^{A_2} \oplus 3 \Gamma^{B_1} \oplus 5 \Gamma^{B_2} \oplus 8 \Gamma^{E_1} \oplus 9 \Gamma^{E_2} \oplus 8 \Gamma^{E_3} . 
	\end{equation}
	At last, we list all determination of the spectroscopic activity of each fundamental level as follows:
	\begin{center}
	{
	\renewcommand{\arraystretch}{1.2}
	\begin{tabular}{c|c|c} \hline
		Irreducible representations & Count & Spectroscopic activity \\ \hline
			$\Gamma^{A_1}$ 	& 6 & Raman \\
			$\Gamma^{A_2}$ 	& 2 & inactive \\
			$\Gamma^{B_1}$ 	& 3 & inactive \\
			$\Gamma^{B_2}$ 	& 5 & infra-red \\
			$\Gamma^{E_1}$ 	& 8 & infra-red	\\ 
			$\Gamma^{E_2}$ 	& 9 & Raman	\\
			$\Gamma^{E_3}$ 	& 8 & Raman \\ \hline
	\end{tabular}
	}
	\end{center}

	\item Firstly, we list the character tables of the  the $\mathscr{C}_{\rm 4v}$ point group.
	\begin{center}
	\captionof{table}{The character table for the $\mathscr{C}_{\rm 4v}$ point group.}\label{table:character_table_of_c4v}
	\begin{tabular}{c|ccccccc} \hline
		$\mathscr{C}_{\rm 4v}$ & $E$ & $2C_4$ & $C_2$ & $2\sigma_{\rm v}$ & $2\sigma_{\rm d}$	&	&	 \\ \hline
		$A_1$	&	1	&	1	&	1	&	1	&	1	&	$z$	&	$x^2+y^2$; $z^2$	\\
		$A_2$	&	1	&	1	&	1	&	-1	&	-1	&	$R_z$	&	\\
		$B_1$	&	1	&	-1	&	1	&	1	&	-1	&	&	$x^2-y^2$	\\
		$B_2$	&	1	&	-1	&	1	&	-1	&	1	&	&	$xy$		\\
		$E$		&	2	&	0	&	-2	&	0	&	0 	& ($x$, $y$); ($R_x$, $R_y$)	&	($xz$, $yz$) 	\\ \hline
	\end{tabular}
	\end{center}	
	
	It is evident that in the $\mathscr{C}_{\rm 2v}$ point group, the $A_1$ and $E$ modes are coincident:
	\begin{align}
		\Gamma^{\rm t} &= \Gamma^{A_1} \oplus \Gamma^E , \\
		\Gamma^{\rm r} &= \Gamma^{A_2} \oplus \Gamma^E , \\
		\Gamma^{\alpha} &= 2 \Gamma^{A_1} \oplus \Gamma^{B_1} \oplus \Gamma^{B_2} \oplus \Gamma^E .
	\end{align}
	
	The character of $\Gamma^0$ is
	\begin{center}
	\begin{tabular}{c|ccccc} \hline
		$R=$		&  $E$ & $2C_4$ & $C_2$ & $2\sigma_{\rm v}$ & $2\sigma_{\rm d}$ \\ \hline
		$\chi^0(C_i)$ & 72 & 0 & 0 & 6 & 6 \\ \hline
	\end{tabular}
	\end{center}
	Solving the system of linear equations like in Problem 7.1, we obtain
	\begin{equation*}
		\Gamma^0 = 12 \Gamma^{A_1} \oplus 6 \Gamma^{A_2} \oplus 9 \Gamma^{B_1} \oplus 9 \Gamma^{B_2} \oplus 18 \Gamma^E .
	\end{equation*}
	Hence, we obtain
	\begin{equation}
		\Gamma^0 = 11 \Gamma^{A_1} \oplus 5 \Gamma^{A_2} \oplus 9 \Gamma^{B_1} \oplus 9 \Gamma^{B_2} \oplus 16 \Gamma^E .
	\end{equation}
	At last, we list all determination of the spectroscopic activity of each fundamental level as follows:
	\begin{center}
	{
	\renewcommand{\arraystretch}{1.2}
	\begin{tabular}{c|c|c} \hline
		Irreducible representations & Count & Spectroscopic activity \\ \hline
			$\Gamma^{A_1}$ 	& 11 & infra-red and Raman \\
			$\Gamma^{A_2}$ 	& 5  & inactive \\
			$\Gamma^{B_1}$ 	& 9  & Raman \\
			$\Gamma^{B_2}$ 	& 9  & Raman \\
			$\Gamma^E$ 		& 16 & infra-red and Raman \\ \hline
	\end{tabular}
	}
	\end{center}
	
	\end{itemize}
	
	Therefore, by inspecting whether both the infra-red and Raman spectra appears at the same frequency, inspectors will distinguish the two crown forms of octachlorocyclooctane.

	\end{solution}
	
	\begin{remark}
	
	\begin{itemize}
	
	\item Although there is no center of inversion ($i$), for the $\mathscr{D}_{\rm 4d}$, the symmetry is high enough that no vibration mode can be both infra-red and Raman active.
	
	\item In fact, because $\mathscr{C}_{\rm 4v}$ is a subgroup of $\mathscr{D}_{\rm 4d}$, their irreducible representations have a relationship. By inspecting the correspondence between two sets of basis functions and rotations, I find the descent path from $\mathscr{D}_{\rm 4d}$ to $\mathscr{C}_{\rm 4v}$, illustrated as the following diagram.
	
	\begin{center}
	\includegraphics[scale=1.0]{../diagrams/chapter_09/descent_table.png}
	\captionof{figure}{Descent paths from $\mathscr{D}_{\rm 4d}$ to $\mathscr{C}_{\rm 4v}$. The number in parentheses following the irreducible representations indicates its reduction coefficient in this problem. Orange lines means that the correspondence are ensured by inspecting the same basis functions or rotations, while purple lines means the correspondence are confirmed by inspecting the characters of relevant irreducible representations.}
	\end{center}
	
	\item The second paragraph of the reference answer in the textbook at the page 294 is totally wrong. From this answer, I think that the author may only care about the vibrations of all $\ce{C}-\ce{Cl}$ bonds. In this way, the answer is
	\[
		\Gamma^{\ce{C}-\ce{Cl}} = 3 \Gamma^{A_1} \oplus \Gamma^{A_2} \oplus 2 \Gamma^{B_1} \oplus 2 \Gamma^{B_2} .
	\]
	It does have 3 $\Gamma^{A_1}$, 1 $\Gamma^{A_2}$ and 2 $\Gamma^{B_2}$. He might think that since $x$ and $y$ are equivalent, thus he only needs to write one as a representative (i.e., only $\mathscr{B_2}$ is written). But in group theory reduction, this is illegal! Once a degenerate state splits, both orbitals must be counted simultaneously.
	
	\end{itemize}		
	
	\end{remark}

	% 9.3
	\begin{problem}
	
	Discuss how the {\it cis} and {\it trans} isomers of $\ce{N2F2}$ can be distinguished by infra-red and Raman measurements.	
	
	\end{problem}
	
	\begin{solution}
	
	\begin{itemize}
	
	\item {\it cis}-$\ce{N2F2}$ has a $C_2$ axis and two $\sigma_{\rm v}$ and thus belongs to the $\mathscr{C}_{\rm 2v}$ point group. Similar to Problem 9.2, first of all, we list the character tables of the $\mathscr{C}_{\rm 2v}$ point group as follows.
	\begin{center}
	\captionof{table}{The character table for the $\mathscr{C}_{\rm 2v}$ point group.}\label{table:character_table_of_c2v}
	\begin{tabular}{c|cccccc} \hline
		$\mathscr{C}_{\rm 2v}$ & $E$ & $C_2$ & $\sigma_{\rm v}(xz)$ & $\sigma_{\rm v}(yz)$ &	&	\\ \hline
		$A_1$	&	1	&	1	&	1	&	1	&	$z$	&	$x^2$; $y^2$; $z^2$	\\
		$A_2$	&	1	&	1	&	-1	&	-1	&	$R_z$	&	$xy$		\\
		$B_1$	&	1	&	-1	&	1	&	-1	&	$x$; $R_y$	&	$xz$	\\
		$B_2$	&	1	&	-1	&	-1	&	1	&	$y$; $R_x$	&	$yz$	\\ \hline
	\end{tabular}
	\end{center}
	In the $\mathscr{C}_{\rm 2v}$ point group, the irreducible representations of the infra-red active modes and Raman active modes have the same irreducible representations $\Gamma^{A_1}$, $\Gamma^{B_1}$ and $\Gamma^{B_2}$:
	\begin{align}
		\Gamma^{\rm t} &= \Gamma^{A_1} \oplus \Gamma^{B_1} \oplus \Gamma^{B_2} , \\
		\Gamma^{\alpha} &= 3 \Gamma^{A_1} \oplus \Gamma^{A_2} \oplus \Gamma^{B_1} \oplus \Gamma^{B_2} .
	\end{align}
	Consequently, {\it cis}-$\ce{N2F2}$ exhibits coincident bands in its infra-red and Raman spectra, appearing at the same vibrational frequencies.
	
	\item {\it trans}-$\ce{N2F2}$ has a $C_2$ axis and a $\sigma_{\rm h}$ and thus belongs to the $\mathscr{C}_{\rm 2h}$ point group. Then, we list the character tables of the $\mathscr{C}_{\rm 2h}$ point group as follows.
	\begin{center}
	\captionof{table}{The character table for the $\mathscr{C}_{\rm 2h}$ point group.}\label{table:character_table_of_c2h}
	\begin{tabular}{c|cccccc} \hline
		$\mathscr{C}_{\rm 2h}$ & $E$ & $C_2$ & $i$ & $\sigma_{\rm h}$ &	&	\\ \hline
		$A_g$	&	1	&	1	&	1	&	1	&	$R_z$	&	$x^2$; $y^2$; $z^2$; $xy$	\\
		$B_g	$	&	1	&	-1	&	1	&	-1	&	$R_x$; $R_y$	&	$xz$; $yz$	\\
		$A_u$	&	1	&	1	&	-1	&	-1	&	$z$		&	\\
		$B_u$	&	1	&	-1	&	-1	&	1	&	$x$; $y$	&	\\ \hline
	\end{tabular}
	\end{center}
	In the $\mathscr{C}_{\rm 2h}$ point group, the irreducible representations of the infra-red active modes and Raman active modes are mutually exclusive:
	\begin{align}
		\Gamma^{\rm t} &= \Gamma^{A_u} \oplus 2 \Gamma^{B_u} , \\
		\Gamma^{\alpha} &= 4 \Gamma^{A_g} \oplus 2 \Gamma^{B_g} .
	\end{align}
	Although there is no center of inversion ($i$), the symmetry is high enough that no vibration mode can be both infra-red and Raman active. In other words, {\it trans}-$\ce{N2F2}$ does not exhibit coincident bands in its infra-red and Raman spectra, appearing at the same vibrational frequencies.
	
	\end{itemize}
	
	\end{solution}
	
	% 9.4
	\begin{problem}
		
	What will be the infra-red and Raman activity of the four fundamental levels of $\ce{CO^{2-}_3}$?
		
	\end{problem}
	
	\begin{solution}
	
	Similar to Problem 9.1, we should solve the current problem in 3 steps.
	
	\begin{enumerate}
	
	\item Determine the point group: $\ce{CO^{2-}_3}$ has a $C_3$ axis and 3 $C_2$ axes perpendicular to this $C_3$ axis. Moreover, it has a $\sigma_{\rm h}$. Therefore, it belongs to the $\mathscr{D}_{\rm 3h}$ point group.
	
	\item Determine the number and symmetry of the vibrational modes. It is clear that from the following character table for the $\mathscr{D}_{\rm 3h}$ point group,
	\begin{align}
		\Gamma^{\rm t} &= \Gamma^{E^\prime} \oplus \Gamma^{A^{\prime\prime}_2} , \\
		\Gamma^{\rm r} &= \Gamma^{A^\prime_2} \oplus \Gamma^{E^{\prime\prime}} , \\
		\Gamma^{\alpha} &= 2 \Gamma^{A^\prime_1} \oplus \Gamma^{E^\prime} \oplus \Gamma^{E^{\prime\prime}} .		
	\end{align}
	\begin{center}
	\captionof{table}{The character table for the $\mathscr{D}_{\rm 3h}$ point group.} \label{table:character_table_of_d3h}
	\begin{tabular}{c|cccccccc}\hline
$\mathscr{D}_{\rm 3h}$ & $E$ & $2C_3$ & $3C_2$ &	$\sigma_{\rm h}$	& $2S_3$ & $3\sigma_{\rm v}$	&	& \\ \hline
		$A^\prime_1$	&	1	&	1	&	1	&	1	&	1	&	1	&	&$x^2+y^2$; $z^2$	\\
		$A^\prime_2$	&	1	&	1	&	-1	&	1	&	1	&	-1	&	$R_z$	&	\\
		$E^\prime$	&	2	&	-1	&	0	&	2	&	-1	&	0	& ($x$, $y$)	&	($x^2-y^2$, $xy$)\\
		$A^{\prime\prime}_1$	&	1	&	1	&	1	&	-1	&	-1	&	-1	&	&	\\
		$A^{\prime\prime}_2$	&	1	&	1	&	-1	&	-1	&	-1	&	1	&	$z$	&	\\
		$E^{\prime\prime}$	&	2	&	-1	&	0	&	-2	&	1	&	0	&	($R_x$, $R_y$)	&	($xz$, $yz$)	\\ \hline
	\end{tabular}
	\end{center}
	
	The character of the $\Gamma^0$ is
	\begin{center}
	\begin{tabular}{c|cccccc} \hline
		$R=$		& $E$ & $2C_3$ & $3C_2$ &	$\sigma_{\rm h}$	& $2S_3$ & $3\sigma_{\rm v}$ \\ \hline
		$\chi^0(C_i)$ & 12 & 0 & -2 & 4 & -2 & 2 \\ \hline
	\end{tabular}
	\end{center}
	Solving the system of linear equations like in Problem 7.1, we obtain
	\begin{equation*}
		\Gamma^0 = \Gamma^{A^\prime_1} \oplus \Gamma^{A^\prime_2} \oplus 3  \Gamma^{E^\prime} \oplus 2 \Gamma^{A^{\prime\prime}_2} \oplus \Gamma^{E^{\prime\prime}} .
	\end{equation*}
	Hence, we obtain
	\begin{equation}
		\Gamma^{\rm v} = \Gamma^{A^\prime_1} \oplus 2 \Gamma^{E^\prime} \oplus \Gamma^{A^{\prime\prime}_2} .
	\end{equation}
	Therefore, we conclude that there are in total 6 vibrational normal modes, and 1, 4, 1 for $\Gamma^{A^\prime_1}$, $\Gamma^{E^\prime}$, and $\Gamma^{A^{\prime\prime}_2}$, respectively.
	
	\item We list all determination of the spectroscopic activity of each fundamental level as follows:
	\begin{center}
	{
	\renewcommand{\arraystretch}{1.2}
	\begin{tabular}{c|c|c} \hline
		Irreducible representations & Count & Spectroscopic activity \\ \hline
			$\Gamma^{A^\prime_1}$	& 1 & Raman \\
			$\Gamma^{E^\prime}$ 		& 2 & Raman and infra-red \\
			$\Gamma^{A^{\prime\prime}_2}$ & 1 & infra-red \\ \hline
	\end{tabular}
	}
	\end{center}
	
	\end{enumerate}
	
	\end{solution}
	
	% 9.5
	\begin{problem}
	
	Determine $\chi^0$ and carry out the reduction of $\Gamma^0$ for the following molecules:
	\begin{enumerate}[label=(\alph*)]
	
	\item $\ce{NH3}$ ($\mathscr{C}_{\rm 3v}$)	,
	
	\item $\ce{XeOF4}$ ($\mathscr{C}_{\rm 4v}$),
	
	\item $\ce{PtCl^{2-}_4}$ ($\mathscr{D}_{\rm 4h}$),
	
	\item {\it trans}-glyoxal ($\mathscr{C}_{\rm 2h}$).
	
	\end{enumerate}
		
	\end{problem}
	
	\begin{solution}
	
	After Problem 9.1 and Problem 9.4, I believe that readers are familiar with the whole solution processes. I tend to demonstrate only the important intermediate results rather than the whole processes.	And whatever, in all issues, we have
	\begin{equation}
		\Gamma^0 = \Gamma^{\rm t,r} \oplus \Gamma^{\rm v} = \Gamma^{\rm t} \oplus \Gamma^{\rm r} \oplus \Gamma^{\rm v} .
	\end{equation}
	
	\begin{enumerate}[label=(\alph*)]
	
	\item The character table for the $\mathscr{C}_{\rm 3v}$ point group is below, and we obtain
	\begin{align}\label{eq:951_t}
		\Gamma^{\rm t} &= \Gamma^{A_1} \oplus \Gamma^E , \\ \label{eq:951_r}
		\Gamma^{\rm r} &= \Gamma^{A_2} \oplus \Gamma^E .
	\end{align}
	
	\begin{center}
	\captionof{table}{The character table for the $\mathscr{C}_{\rm 3v}$ point group.} \label{table:character_table_of_c3v}
	\begin{tabular}{c|ccccc}\hline
$\mathscr{C}_{\rm 3v}$ & $E$ & $2C_3$ &	$3\sigma_{\rm v}$	&	& \\ \hline
		$A_1$	&	1	&	1	&	1	&	$z$		&	$x^2+y^2$; $z^2$\\
		$A_2$	&	1	&	1	&	-1	&	$R_z$	&		\\
		$E$		&	2	&	-1	&	0	&($x$, $y$); ($R_x$, $R_y$)	&	($x^2-y^2$, $xy$); ($xz$; $yz$)\\ \hline
	\end{tabular}
	\end{center}
	
	The character of the $\Gamma^0$ is
	\begin{center}
	\begin{tabular}{c|ccc} \hline
		$R=$		& $E$ & $2C_3$ &	$3\sigma_{\rm v}$ \\ \hline
		$\chi^0(C_i)$ & 12 & 0 & 2 \\ \hline
	\end{tabular}
	\end{center}
	Solving the system of linear equations like in Problem 7.1, we obtain
	\begin{equation*}\label{eq:951_0}
		\Gamma^0 = 3 \Gamma^{A_1} \oplus \Gamma^{A_2} \oplus 4 \Gamma^E .
	\end{equation*}
	Hence, we obtain
	\begin{equation}\label{eq:951_v}
		\Gamma^{\rm v} = 2 \Gamma^{A_1} \oplus 2 \Gamma^E .
	\end{equation}
	Therefore, our result are \Eqref{eq:951_t}, \Eqref{eq:951_r}, \Eqref{eq:951_v} and \Eqref{eq:951_v}.
	
	\item The character table for the $\mathscr{C}_{\rm 4v}$ point group can be seen in \Tableref{table:character_table_of_c4v} and we obtain
	\begin{align}\label{eq:952_t}
		\Gamma^{\rm t} &= \Gamma^{A_1} \oplus \Gamma^E , \\ \label{eq:952_r}
		\Gamma^{\rm r} &= \Gamma^{A_2} \oplus \Gamma^E .
	\end{align}
	
	The character of the $\Gamma^0$ is
	\begin{center}
	\begin{tabular}{c|ccccc} \hline
		$R=$		& $E$ & $2C_4$ & $C_2$ & $2\sigma_{\rm v}$ & $2\sigma_{\rm d}$ \\ \hline
		$\chi^0(C_i)$ & 18 & 2 & -2 & 4 & 2 \\ \hline
	\end{tabular}
	\end{center}
	Solving the system of linear equations like in Problem 7.1, we obtain
	\begin{equation}\label{eq:952_0}
		\Gamma^0 = 4 \Gamma^{A_1} \oplus \Gamma^{A_2} \oplus 2 \Gamma^{B_1} \oplus \Gamma^{B_2} \oplus 5 \Gamma^E .
	\end{equation}
	Hence, we obtain
	\begin{equation}\label{eq:952_v}
		\Gamma^{\rm v} = 3 \Gamma^{A_1} \oplus 2 \Gamma^{B_1} \oplus \Gamma^{B_2} \oplus 5 \Gamma^E .
	\end{equation}
	Therefore, our result are \Eqref{eq:952_t}, \Eqref{eq:952_r}, \Eqref{eq:952_0}, and \Eqref{eq:952_v}.
	
	\item The character table for the $\mathscr{D}_{\rm 4h}$ point group is below, and we obtain
	\begin{align}\label{eq:953_t}
		\Gamma^{\rm t} &= \Gamma^{A_{2u}} \oplus \Gamma^{E_u} , \\ \label{eq:953_r}
		\Gamma^{\rm r} &= \Gamma^{A_{2g}} \oplus \Gamma^{E_g} .
	\end{align}
	
	\begin{center}
	\captionof{table}{The character table for the $\mathscr{D}_{\rm 4h}$ point group.} \label{table:character_table_of_d4h}
	\begin{tabular}{c|cccccccccccc}\hline
	$\mathscr{D}_{\rm 4h}$ & $E$ & $2C_4$ &	$C_2$	& $2C^\prime_2$	&	$2C^{\prime\prime}_2$	&	$i$	&	$2S_4$	&	$\sigma_{h}$	&	$2\sigma_{v}$ &	$2\sigma_{d}$	&		&\\ \hline
		$A_{1g}$	&	1	&	1	&	1	&	1	&	1	&	1	&	1	&	1	&	1	&	1	&		&	$x^2+y^2$; $z^2$\\
		$A_{2g}$	&	1	&	1	&	1	&	-1	&	-1	&	1	&	1	&	1	&	-1	&	-1	& $R_z$	&	\\
		$B_{1g}$	&	1	&	-1	&	1	&	1	&	-1	&	1	&	-1	&	1	&	1	&	-1	&		&	$x^2-y^2$\\
		$B_{2g}$ 	&	1	&	-1	&	1	&	-1	&	1	&	1	&	-1	&	1	&	-1	&	1	&		&	$xy$	\\
		$E_g$ 		&	2	&	0	&	-2	&	0	&	0	&	2	&	0	&	-2	&	0	&	0	& ($R_x$, $R_y$) & ($xz$, $yz$)\\ 
		$A_{1u}$	&	1	&	1	&	1	&	1	&	1	&	-1	&	-1	&	-1	&	-1	&	-1	&		&	\\
		$A_{2u}$	&	1	&	1	&	1	&	-1	&	-1	&	-1	&	-1	&	-1	&	1	&	1	&	$z$	&	\\
		$B_{1u}$	&	1	&	-1	&	1	&	1	&	-1	&	-1	&	1	&	-1	&	-1	&	1	&		&	\\
		$B_{2u}$ 	&	1	&	-1	&	1	&	-1	&	1	&	-1	&	1	&	-1	&	1	&	-1	&		&	\\
		$E_u$ 		&	2	&	0	&	-2	&	0	&	0	&	-2	&	0	&	2	&	0	&	0	& ($x$, $y$)	&\\ \hline
	\end{tabular}
	\end{center}
	
	The character of the $\Gamma^0$ is
	\begin{center}
	\begin{tabular}{c|cccccccccc} \hline
		$R=$		&	$E$ & $2C_4$ &	$C_2$	& $2C^\prime_2$	&	$2C^{\prime\prime}_2$	&	$i$	&	$2S_4$	&	$\sigma_{h}$	&	$2\sigma_{\rm v}$ &	$2\sigma_{\rm d}$  \\ \hline
		$\chi^0(C_i)$ & 15 & 1 & -1 & -3 & -1 & -3 & -1 & 5 & 3 & 1 \\ \hline
	\end{tabular}
	\end{center}
	Solving the system of linear equations like in Problem 7.1, we obtain
	\begin{equation}\label{eq:953_0}
		\Gamma^0 = \Gamma^{A_{1g}} \oplus \Gamma^{A_{2g}} \oplus \Gamma^{B_{1g}} \oplus \Gamma^{B_{2g}} \oplus \Gamma^{E_g} \oplus 2 \Gamma^{A_{2u}} \oplus \Gamma^{B_{2u}} \oplus 3 \Gamma^{E_u} .
	\end{equation}
	Hence, we obtain
	\begin{equation}\label{eq:953_v}
		\Gamma^{\rm v} = \Gamma^{A_{1g}} \oplus \Gamma^{B_{1g}} \oplus \Gamma^{B_{2g}} \oplus \Gamma^{A_{2u}} \oplus \Gamma^{B_{2u}} \oplus 2 \Gamma^{E_u} .
	\end{equation}
	Therefore, our result are \Eqref{eq:953_t}, \Eqref{eq:953_r}, \Eqref{eq:953_0}, and \Eqref{eq:953_v}.
	
	\item The character table for the $\mathscr{C}_{\rm 2h}$ point group can be seen in \Tableref{table:character_table_of_c2h} and we obtain
	\begin{align}\label{eq:954_t}
		\Gamma^{\rm t} &= \Gamma^{A_u} \oplus 2 \Gamma^{B_u} , \\ \label{eq:954_r}
		\Gamma^{\rm r} &= \Gamma^{A_g} \oplus 2 \Gamma^{B_g} .
	\end{align}
	
	The character of the $\Gamma^0$ is
	\begin{center}
	\begin{tabular}{c|cccc} \hline
		$R=$		& $E$ & $C_2$ & $i$ & $\sigma_{\rm h}$ \\ \hline
		$\chi^0(C_i)$ & 18 & 0 & 0 & 6 \\ \hline
	\end{tabular}
	\end{center}
	Solving the system of linear equations like in Problem 7.1, we obtain
	\begin{equation}\label{eq:954_0}
		\Gamma^0 = 6 \Gamma^{A_1} \oplus 3 \Gamma^{A_2} \oplus 3 \Gamma^{B_1} \oplus 6 \Gamma^{B_2} .
	\end{equation}
	Hence, we obtain
	\begin{equation}\label{eq:954_v}
		\Gamma^{\rm v} = 5 \Gamma^{A_1} \oplus \Gamma^{A_2} \oplus 2 \Gamma^{B_1} \oplus 4 \Gamma^{B_2} .
	\end{equation}
	Therefore, our result are \Eqref{eq:954_t}, \Eqref{eq:954_r}, \Eqref{eq:954_0}, and \Eqref{eq:954_v}.

	\end{enumerate}	
	
	\end{solution}	

\end{document}