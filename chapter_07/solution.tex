\documentclass[a4paper]{book}

\usepackage{afterpage}
\usepackage[hypcap=false]{caption}
\usepackage{enumitem}	% 定制enumerate标号
\usepackage{geometry}
\geometry{%
	left=2cm,%
	right=2cm,%
	top=2cm,%
	bottom=2cm,%
	bindingoffset=0cm
}
\usepackage{hyperref}
\hypersetup{
    colorlinks=true,            %链接颜色
    linkcolor=blue,             %内部链接
    filecolor=magenta,          %本地文档
    urlcolor=cyan,              %网址链接
    pdftitle={Overleaf Example},
    pdfpagemode=FullScreen,
}
\usepackage[none]{hyphenat}	% 阻止长单词分在两行
\usepackage{longtable}
\usepackage{mathrsfs}	% 提供\mathscr字体
\usepackage[version=4]{mhchem}
\usepackage{multirow}
\usepackage{subcaption}

\RequirePackage[many]{tcolorbox}
\tcbset{
    boxed title style={colback=magenta},
	breakable,
	enhanced,
	sharp corners,
	attach boxed title to top left={yshift=-\tcboxedtitleheight,  yshifttext=-.75\baselineskip},
	boxed title style={boxsep=1pt,sharp corners},
    fonttitle=\bfseries\sffamily,
}

\definecolor{skyblue}{rgb}{0.54, 0.81, 0.94}

\newtcolorbox[auto counter, number within=chapter, number format=\arabic]{exercise}[1][]{
    title={Exercise~\thetcbcounter},
    colframe=skyblue,
    colback=skyblue!12!white,
    boxed title style={colback=skyblue},
    overlay unbroken and first={
        \node[below right,font=\small,color=skyblue,text width=.8\linewidth]
        at (title.north east) {#1};
    }
}

\newtcolorbox[auto counter, number within=chapter, number format=\arabic]{solution}[1][]{
%    top=2ex,
%    boxrule=0pt,
%    leftrule=1.4pt,
    title={Solution~\thetcbcounter},
    colframe=teal!60!green,
    colback=green!12!white,
    boxed title style={colback=teal!60!green},
    overlay unbroken and first={
        \node[below right,font=\small,color=red,text width=.8\linewidth]
        at (title.north east) {#1};
    }
}

\newcommand{\AO}{{\rm AO}}
\newcommand{\Heff}{H^{\rm eff,\pi}}
\newcommand{\Hp}{H^\prime}
\newcommand{\Sp}{S^\prime}
\newcommand{\RRR}{{\rm R}^3}
\newcommand\Figref[1]{Fig \ref{#1}}
\newcommand\Tableref[1]{Table \ref{#1}}
\newcommand{\orb}[1]{{\rm #1}}
\newcommand{\orbs}{\orb{s}}
\newcommand{\orbp}{\orb{p}}
\newcommand{\orbd}{\orb{d}}
\newcommand{\orbf}{\orb{f}}

\allowdisplaybreaks

\begin{document}

	\setcounter{chapter}{7}
	
	% 7.1
	\begin{exercise}
		
		Given the characters $\chi$ of a reducible representation $\Gamma$ of the indicated point group $\mathscr{G}$ for the various classes of $\mathscr{G}$ in the order in which these classes appear in the character table, find the number of times irreducible representation occurs in $\Gamma$.
		\begin{enumerate}[label=(\alph*)]
		
		\item $\mathscr{C}_{\rm 2v}$ $\chi = 4, -2, 0, -2$,
		
		\item $\mathscr{C}_{\rm 3h}$ $\chi = 4, 1, 1, 2, -1, -1$,
		
		\item $\mathscr{D}_{\rm 4d}$ $\chi = 6, 0, -2, 0, -2, 0, 0$,
		
		\item $\mathscr{O}_{\rm h}$ $\chi = 15, 0, -1, 1, 1, -3, 0, 5, -1 , 3$.
		
		\end{enumerate}				
		
	\end{exercise}

	\begin{solution}
	
	\begin{enumerate}[label=(\alph*)]
	
	\item 1
	
	\begin{center}
	\begin{tabular}{c|cccc} \hline
		$\mathscr{C}_{\rm 2v}$ & $E$ & $C_2$ & $\sigma_{\rm v}(xz)$ & $\sigma_{\rm v}(yz)$ \\ \hline
		$A_1$	&	1	&	1	&	1	&	1	\\
		$A_2$	&	1	&	1	&	-1	&	-1	\\
		$B_1$	&	1	&	-1	&	1	&	-1	\\
		$B_2$	&	1	&	-1	&	-1	&	1	\\ \hline
	\end{tabular}
	\end{center}
	
	\item 2
	
	\begin{center}
	\begin{tabular}{c|cccccc} \hline
		$\mathscr{C}_{\rm 3h}$ & $E$ & $C_3$ & $C^2_3$ & $\sigma_{\rm h}$ & $S_3$ & $S^5_3$ \\ \hline
		$A^\prime$	&	1	&	1	&	1	&	1	&	1	&	1	\\
		\multirow{2}*{$E^\prime$}	&	1	&	$\varepsilon$	&	$\varepsilon^*$	&	1	&	$\varepsilon$	&	$\varepsilon^*$	\\
					&	1	&	$\varepsilon^*$	&	$\varepsilon$	&	1	&	$\varepsilon^*$	&	$\varepsilon$	\\
		$A^{\prime\prime}$	&	1	&	1	&	1	&	-1	&	-1	&	-1	\\
		\multirow{2}*{$E^{\prime\prime}$}	&	1	&	$\varepsilon$	&	$\varepsilon^*$	&	-1	&	$-\varepsilon$	&	$-\varepsilon^*$	\\
					&	1	&	$\varepsilon^*$	&	$\varepsilon$	&	-1	&	$-\varepsilon^*$	&	$-\varepsilon$	\\ \hline
	\end{tabular}
	\end{center}
	
	\item 3
	
	\begin{center}
	\begin{tabular}{c|ccccccc} \hline
		$\mathscr{D}_{\rm 4d}$ & $E$ & $2S_8$ & $2C_4$ & $2S^3_8$ & $C_2$ & $4C^\prime_2$ & $4 \sigma_{\rm d}$ \\ \hline
		$A_1$	&	1	&	1	&	1	&	1	&	1	&	1	&	1	\\
		$A_2$	&	1	&	1	&	1	&	1	&	1	&	-1	&	-1	\\
		$B_1$	&	1	&	-1	&	1	&	-1	&	1	&	1	&	-1	\\
		$B_2$	&	1	&	-1	&	1	&	-1	&	1	&	-1	&	1	\\
		$E_1$	&	2	& $\sqrt{2}$	& 0	& $-\sqrt{2}$ & -2 & 0 	&	0	\\
		$E_2$	&	2	& 	0	& 	-2	& 	0 	& 	2 	& 	0 	&	0	\\
		$E_3$	&	2	& $-\sqrt{2}$ & 0 & $\sqrt{2}$ & -2 & 0 	&	0	\\ \hline
	\end{tabular}
	\end{center}
	
	\item 4
	
	\begin{center}
	\begin{tabular}{c|cccccccccc} \hline
		$\mathscr{O}_{\rm h}$ & $E$ & $8C_3$ & $3C_2$ & $6C_4$ & $6C^\prime_2$ & $i$ & $8 S_6$ & $3 \sigma_{\rm h}$ & $6S_4$ & $6 \sigma_{\rm d}$ \\ \hline
		$A_{1g}$	&	1	&	1	&	1	&	1	&	1	&	1	&	1	&	1	&	1	&	1	\\
		$A_{2g}$	&	1	&	1	&	1	&	-1	&	-1	&	1	&	1	&	1	&	-1	&	-1	\\
		$E_g$	&	2	&	-1	&	2	&	0	&	0	&	2	&	-1	&	2	&	0	&	0	\\
		$T_{1g}$ &	3	&	0	&	-1	&	1	&	-1	&	3	&	0	&	-1	&	1	&	-1	\\
		$T_{2g}$ &	3	&	0	&	-1	&	-1	&	1	&	3	&	0	&	-1	&	-1	&	1	\\
		$A_{1u}$	&	1	&	1	&	1	&	1	&	1	&	-1	&	-1	&	-1	&	-1	&	-1	\\
		$A_{2u}$	&	1	&	1	&	1	&	-1	&	-1	&	-1	&	-1	&	-1	&	1	&	1	\\
		$E_g$	&	2	&	-1	&	2	&	0	&	0	&	-2	&	1	&	-2	&	0	&	0	\\
		$T_{1u}$	 & 	3	&	0	&	-1	&	1	&	-1	&	-3	&	0	&	1	&	-1	&	1	\\
		$T_{2u}$ &	3	&	0	&	-1	&	-1	&	1	&	-3	&	0	&	1	&	1	&	-1	\\ \hline
	\end{tabular}
	\end{center}
	
	\end{enumerate}
	
	\end{solution}
	
	% 7.2
	\begin{exercise}
	
	Consider the four functions of Problem 5.2 which form a basis for a reducible representation $\Gamma$ of $\mathscr{D}_4$. Using projection operators find the orthonormal basis functions which reduce $\Gamma$. Assume $(f_i, f_j) = \delta_{ij}$.	
		
	\end{exercise}

	\begin{solution}

	\begin{center}
		\setlength{\abovecaptionskip}{0em}
		\captionof{table}{The character table for the $\mathscr{D}_{\rm 4}$ point group.}\label{tab:chatab_3}
		\begin{tabular}{cccccc}\hline
	$\mathscr{D}_{\rm 4}$ & $E$ & $2C_4$ &	$C_2$	& $2C^\prime_2$ & $2C^{\prime\prime}_2$ \\ \hline
			$A_1$	&	1	&	1	&	1	&	1	&	1	\\
			$A_1$	&	1	&	1	&	1	&	-1	&	-1	\\
			$B_1$	&	1	&	-1	&	1	&	1	&	-1	\\
			$B_2$	&	1	&	-1	&	1	&	-1	&	1	\\
			$E$ 	&	2	&	0	&	-2	&	0	&	0\\ \hline
		\end{tabular}
		\end{center}

	\end{solution}

	% 7.3
	\begin{exercise}
	
	Show that the characters of $\mathscr{C}_{\rm 4v}$ obey the orthogonality rules of eqns (7.3-5) and (A.7-3.10).
	
	\end{exercise}
	
	\begin{solution}
		
	\end{solution}
	
	% 7.4
	\begin{exercise}
		
		How many times does each irreducible representation of the $\mathscr{C}_{\rm 2v}$ point group occur in the nine-dimentional representation found in Problem 5.3?
		
	\end{exercise}
	
	\begin{solution}
	
	\end{solution}
	
	% 7.5
	\begin{exercise}
		
		Consider the group whose group table is
		\begin{center}
		\begin{tabular}{c|cccc}\hline
				&	$E$	&	$A$	&	$B$	&	$C$	\\ \hline
			$E$	&	$E$	&	$A$	&	$B$	&	$C$	\\
			$A$	&	$A$	&	$C$	&	$E$	&	$B$	\\
			$B$	&	$B$	&	$E$	&	$C$	&	$E$	\\
			$C$	&	$C$	&	$B$	&	$A$	&	$A$	\\ \hline
		\end{tabular}
		\end{center}
		write out the matrices and characters for the regular representation of this group.

	\end{exercise}
	
	\begin{solution}
	
	\end{solution}
	
	% 7.6
	\begin{exercise}
		
		Determine the irreducible representation to which the following real orbitals belong for the indicated point group:
		\begin{enumerate}
		
		\item $\orbp_1$, $\orbp_2$, $\orbp_3$ in $\mathscr{D}_4$ and $\mathscr{D}_{\rm 2h}$ ,
		
		\item $\orbd_1$, $\orbd_2$, $\orbd_3$, $\orbd_4$, $\orbd_5$ in $\mathscr{O}_{\rm h}$,
		
		\item $\orbd_1$, $\orbd_2$, $\orbd_3$, $\orbd_4$, $\orbd_5$ in $\mathscr{D}_{\rm 3h}$,
		
		\item $\orbd_1$, $\orbd_2$, $\orbd_3$, $\orbd_4$, $\orbd_5$ in $\mathscr{T}_{\rm d}$.
		
		\end{enumerate}				
		
	\end{exercise}
	
	\begin{solution}
	
	\end{solution}

\end{document}