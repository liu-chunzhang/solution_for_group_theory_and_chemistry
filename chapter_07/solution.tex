\documentclass[a4paper]{book}

\usepackage{afterpage}
\usepackage[hypcap=false]{caption}
\usepackage{enumitem}	% 定制enumerate标号
\usepackage{geometry}
\geometry{%
	left=2cm,%
	right=2cm,%
	top=2cm,%
	bottom=2cm,%
	bindingoffset=0cm
}
\usepackage{hyperref}
\hypersetup{
    colorlinks=true,            %链接颜色
    linkcolor=blue,             %内部链接
    filecolor=magenta,          %本地文档
    urlcolor=cyan,              %网址链接
    pdftitle={Overleaf Example},
    pdfpagemode=FullScreen,
}
\usepackage[none]{hyphenat}	% 阻止长单词分在两行
\usepackage{longtable}
\usepackage{mathrsfs}	% 提供\mathscr字体
\usepackage[version=4]{mhchem}
\usepackage{multirow}
\usepackage{subcaption}

\RequirePackage[many]{tcolorbox}
\tcbset{
    boxed title style={colback=magenta},
	breakable,
	enhanced,
	sharp corners,
	attach boxed title to top left={yshift=-\tcboxedtitleheight,  yshifttext=-.75\baselineskip},
	boxed title style={boxsep=1pt,sharp corners},
    fonttitle=\bfseries\sffamily,
}

\definecolor{skyblue}{rgb}{0.54, 0.81, 0.94}

\newtcolorbox[auto counter, number within=chapter, number format=\arabic]{exercise}[1][]{
    title={Exercise~\thetcbcounter},
    colframe=skyblue,
    colback=skyblue!12!white,
    boxed title style={colback=skyblue},
    overlay unbroken and first={
        \node[below right,font=\small,color=skyblue,text width=.8\linewidth]
        at (title.north east) {#1};
    }
}

\newtcolorbox[auto counter, number within=chapter, number format=\arabic]{solution}[1][]{
%    top=2ex,
%    boxrule=0pt,
%    leftrule=1.4pt,
    title={Solution~\thetcbcounter},
    colframe=teal!60!green,
    colback=green!12!white,
    boxed title style={colback=teal!60!green},
    overlay unbroken and first={
        \node[below right,font=\small,color=red,text width=.8\linewidth]
        at (title.north east) {#1};
    }
}

\newcommand{\AO}{{\rm AO}}
\newcommand{\Heff}{H^{\rm eff,\pi}}
\newcommand{\Hp}{H^\prime}
\newcommand{\Sp}{S^\prime}
\newcommand{\RRR}{{\rm R}^3}
\newcommand\Figref[1]{Fig \ref{#1}}
\newcommand\Tableref[1]{Table \ref{#1}}
\newcommand{\orb}[1]{{\rm #1}}
\newcommand{\orbs}{\orb{s}}
\newcommand{\orbp}{\orb{p}}
\newcommand{\orbd}{\orb{d}}
\newcommand{\orbf}{\orb{f}}

\allowdisplaybreaks

\begin{document}

	\setcounter{chapter}{7}
	
	% 7.1
	\begin{exercise}
		
		Given the characters $\chi$ of a reducible representation $\Gamma$ of the indicated point group $\mathscr{G}$ for the various classes of $\mathscr{G}$ in the order in which these classes appear in the character table, find the number of times irreducible representation occurs in $\Gamma$.
		\begin{enumerate}
		
		\item $\mathscr{C}_{\rm 2v}$ $\chi = 4, -2, 0, -2$,
		
		\item $\mathscr{C}_{\rm 3h}$ $\chi = 4, 1, 1, 2, -1, -1$,
		
		\item $\mathscr{D}_{\rm 4d}$ $\chi = 6, 0, -2, 0, -2, 0, 0$,
		
		\item $\mathscr{O}_{\rm h}$ $\chi = 15, 0, -1, 1, 1, -3, 0, 5, -1 , 3$.
		
		\end{enumerate}				
		
	\end{exercise}

	\begin{solution}
	
	\end{solution}
	
	% 7.2
	\begin{exercise}
	
	Consider the four functions of Problem 5.2 which form a basis for a reducible representation $\Gamma$ of $\mathscr{D}_4$. Using projection operators find the orthonormal basis functions which reduce $\Gamma$. Assume $(f_i, f_j) = \delta_{ij}$.	
		
	\end{exercise}

	\begin{solution}

	\end{solution}

	% 7.3
	\begin{exercise}
	
	Show that the characters of $\mathscr{C}_{\rm 4v}$ obey the orthogonality rules of eqns (7.3-5) and (A.7-3.10).
	
	\end{exercise}
	
	\begin{solution}
		
	\end{solution}
	
	% 7.4
	\begin{exercise}
		
		How many times does each irreducible representation of the $\mathscr{C}_{\rm 2v}$ point group occur in the nine-dimentional representation found in Problem 5.3?
		
	\end{exercise}
	
	\begin{solution}
	
	\end{solution}
	
	% 7.5
	\begin{exercise}
		
		Consider the group whose group table is
		\begin{tabular}{c|cccc}\hline
				&	$E$	&	$A$	&	$B$	&	$C$	\\ \hline
			$E$	&	$E$	&	$A$	&	$B$	&	$C$	\\
			$A$	&	$A$	&	$C$	&	$E$	&	$B$	\\
			$B$	&	$B$	&	$E$	&	$C$	&	$E$	\\
			$C$	&	$C$	&	$B$	&	$A$	&	$A$	\\ \hline
		\end{tabular}

	\end{exercise}
	
	\begin{solution}
	
	\end{solution}
	
	% 7.6
	\begin{exercise}
		
		Determine the irreducible representation to which the following real orbitals belong for the indicated point group:
		\begin{enumerate}
		
		\item $\orbp_1$, $\orbp_2$, $\orbp_3$ in $\mathscr{D}_4$ and $\mathscr{D}_{\rm 2h}$ ,
		
		\item $\orbd_1$, $\orbd_2$, $\orbd_3$, $\orbd_4$, $\orbd_5$ in $\mathscr{O}_{\rm h}$,
		
		\item $\orbd_1$, $\orbd_2$, $\orbd_3$, $\orbd_4$, $\orbd_5$ in $\mathscr{D}_{\rm 3h}$,
		
		\item $\orbd_1$, $\orbd_2$, $\orbd_3$, $\orbd_4$, $\orbd_5$ in $\mathscr{T}_{\rm d}$.
		
		\end{enumerate}				
		
	\end{exercise}
	
	\begin{solution}
	
	\end{solution}

\end{document}