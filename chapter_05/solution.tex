\documentclass[a4paper]{book}

\usepackage{afterpage}
\usepackage[hypcap=false]{caption}
\usepackage{cellspace}
\setlength\cellspacetoplimit{5pt}    % 设置单元格顶部最小间距
\setlength\cellspacebottomlimit{5pt} % 设置单元格底部最小间距
\usepackage{enumitem}	% 定制enumerate标号
\usepackage{geometry}
\geometry{%
	left=2cm,%
	right=2cm,%
	top=2cm,%
	bottom=2cm,%
	bindingoffset=0cm
}
\usepackage{hyperref}
\hypersetup{
    colorlinks=true,            %链接颜色
    linkcolor=blue,             %内部链接
    filecolor=magenta,          %本地文档
    urlcolor=cyan,              %网址链接
    pdftitle={Overleaf Example},
    pdfpagemode=FullScreen,
}
\usepackage[none]{hyphenat}	% 阻止长单词分在两行
\usepackage{longtable}
\usepackage{mathrsfs}	% 提供\mathscr字体
\usepackage[version=4]{mhchem}
\usepackage{multirow}
\usepackage{subcaption}
\usepackage{titlesec}

\RequirePackage[many]{tcolorbox}
\tcbset{
    boxed title style={colback=magenta},
	breakable,
	enhanced,
	sharp corners,
	attach boxed title to top left={yshift=-\tcboxedtitleheight,  yshifttext=-.75\baselineskip},
	boxed title style={boxsep=1pt,sharp corners},
    fonttitle=\bfseries\sffamily,
}

\definecolor{skyblue}{rgb}{0.54, 0.81, 0.94}

\newtcolorbox[auto counter, number within=chapter, number format=\arabic]{exercise}[1][]{
    title={Exercise~\thetcbcounter},
    colframe=skyblue,
    colback=skyblue!12!white,
    boxed title style={colback=skyblue},
    overlay unbroken and first={
        \node[below right,font=\small,color=skyblue,text width=.8\linewidth]
        at (title.north east) {#1};
    }
}

\newtcolorbox[auto counter, number within=chapter, number format=\arabic]{solution}[1][]{
%    top=2ex,
%    boxrule=0pt,
%    leftrule=1.4pt,
    title={Solution~\thetcbcounter},
    colframe=teal!60!green,
    colback=green!12!white,
    boxed title style={colback=teal!60!green},
    overlay unbroken and first={
        \node[below right,font=\small,color=red,text width=.8\linewidth]
        at (title.north east) {#1};
    }
}

\newtcolorbox{remark}[1][]{
    title={Remark},
    colframe=yellow!45!orange,
    colback=yellow!45!white,
    coltitle=white,
    boxed title style={colback=yellow!45!orange},
    overlay unbroken and first={
        \node[below right,font=\small,color=white,text width=.8\linewidth]
        at (title.north east) {#1};
    }
}

\newcommand{\AO}{{\rm AO}}
\newcommand{\Heff}{H^{\rm eff,\pi}}
\newcommand{\Hp}{H^\prime}
\newcommand{\Sp}{S^\prime}
\newcommand{\RRR}{{\rm R}^3}
\newcommand\Figref[1]{Fig \ref{#1}}
\newcommand\Tableref[1]{Table \ref{#1}}
\newcommand{\orb}[1]{{\rm #1}}
\newcommand{\orbs}{\orb{s}}
\newcommand{\orbp}{\orb{p}}
\newcommand{\orbd}{\orb{d}}
\newcommand{\orbf}{\orb{f}}

\allowdisplaybreaks

\titleformat{\chapter}[display]
  {\bfseries\Large}
  {\filright\MakeUppercase{\chaptertitlename} \Huge\thechapter}
  {1ex}
  {\titlerule\vspace{1ex}\filleft}
  [\vspace{1ex}\titlerule]

\begin{document}

	\setcounter{chapter}{4}
	
	\chapter{Matrix Representations}
	
	% 5.1
	\begin{exercise}
	
	Consider the following planar symmetric figure.
	
	\begin{center}
		\begin{minipage}[t]{0.6\linewidth}
		\centering
		\setlength{\abovecaptionskip}{0.5em}
		\includegraphics[scale=1.2]{../diagrams/chapter_05/bowtie_in_exercises.png}
		\end{minipage}\label{fig:bowtie_in_exercises}
	\end{center}	
	
	\begin{enumerate}[label=(\alph*)]
	
	\item Determine the distinct symmetry operations which take it into itself; construct the group multiplication table for these operations, and identify the point group to which this figure belongs.
	
	\item Find a set of two-dimensional matrices which are in one-to-one correspondence with the above symmetry operations, and verify that they have the same group multiplication table as the symmetry operations.
	
	\end{enumerate}
	
	\end{exercise}

	\begin{solution}
	
	\begin{enumerate}[label=(\alph*)]
	
	\item It is easy to see that the bowtie-like planar pattern has exactly 4 symmetry operations, which are represented by their corresponding symmetry elements.
	
		\begin{itemize}
	
		\item $E$: The `do nothing' operation.
		
		\item $C_2(z)$: A $180^\circ$ rotation about the $z$-axis (the axis passing through the center perpendicular to the screen).
		
		\item $\sigma_{\rm v}(xz)$: A mirror plane passing through the horizontal axis of the figure.
		
		\item $\sigma_{\rm v}(yz)$: A mirror plane passing through the vertical center point, bisecting the bowtie.
	
		\end{itemize}
		
	Due to the $C_2$ axis and 2 $\sigma_{\rm v}$, the bowtie pattern belongs to the $\mathscr{C}_{\rm 2v}$ point group. Moreover, the group multiplication table for these operations can be seen in \Tableref{table:group_multiplication_table_of_c2v}.
	
	\begin{center}
	\captionof{table}{Group multiplication table for $\mathscr{C}_{\rm 2v}$.}\label{table:group_multiplication_table_of_c2v}
	\begin{tabular}{ccccc} \hline
		$R$	&	$E$	&	$C_2(z)$	&	$\sigma_{\rm v}(xz)$	&	$\sigma_{\rm v}(yz)$ \\ \hline
		$E$	&	$E$	&	$C_2(z)$	&	$\sigma_{\rm v}(xz)$	&	$\sigma_{\rm v}(yz)$ \\
	$C_2(z)$ & $C_2(z)$ & $E$ & $\sigma_{\rm v}(yz)$ & $\sigma_{\rm v}(xz)$  \\
	$\sigma_{\rm v}(xz)$ & $\sigma_{\rm v}(xz)$ & $\sigma_{\rm v}(yz)$ & $E$ & $C_2$ \\
	$\sigma_{\rm v}(yz)$ & $\sigma_{\rm v}(yz)$ & $\sigma_{\rm v}(xz)$ & $C_2$ & $E$ \\ \hline
	\end{tabular}		
	\end{center}
	
	\item We define non-collinear ${\bf e}_1$ and ${\bf e}_2$ as a basis, illustrated in \Figref{fig:bowtie_in_solutions}.
	\begin{center}
		\begin{minipage}[t]{0.6\linewidth}
		\centering
		\setlength{\abovecaptionskip}{0.5em}
		\includegraphics[scale=1.2]{../diagrams/chapter_05/bowtie_in_solutions.png}
		\end{minipage}
		\captionof{figure}{Definition of basis vectors ${\bf e}_1$ and ${\bf e}_2$ in the bowtie pattern, originating from $O$ to $A$ and $B$ respectively.}\label{fig:bowtie_in_solutions}
	\end{center}	
	
	 The table of the transformation of ${\bf e}_i$ under $\boldsymbol{O}_R$ for the bowtie pattern can be seen in \Tableref{table:transformation_table_of_bowtie}.
	\begin{center}
	\captionof{table}{Transformation of ${\bf e}_i$ under $\boldsymbol{O}_R$ for the bowtie pattern.}\label{table:transformation_table_of_bowtie}
	\begin{tabular}{c|cccc}\hline
		$R =$ & 	$E$		& $C_2$ & $\sigma_{xz}$ 	& $\sigma_{yz}$ \\ \hline 
		${\bf e}_1$ &	${\bf e}_1$	& $-{\bf e}_1$	&	${\bf e}_2$	&	$-{\bf e}_2$		\\
		${\bf e}_2$ &	${\bf e}_2$	& $-{\bf e}_2$	&	${\bf e}_1$	&	$-{\bf e}_1$		\\ \hline
	\end{tabular}
	\end{center}
	
	Under this basis, a set of two-dimentional matrices which are in one-to-one correspondence with the above symmetry operations can be constructed in \Tableref{table:matrix_representation_of_bowtie}.
	\begin{center}
	\captionof{table}{A set of two-dimentional matrices which are in one-to-one correspondence with $\mathscr{C}_{\rm 2v}$'s symmetry operations.}\label{table:matrix_representation_of_bowtie}
	\begin{tabular}{c|ScScScSc} \hline
		$R$	&	$E$	&	$C_2$	&	$\sigma_{xz}$	&	$\sigma_{yz}$	\\ \hline
		$D(R)$	& $\begin{pmatrix}
			1 & 0 \\
			0 & 1
		\end{pmatrix}$ & $\begin{pmatrix}
			-1 & 0 \\
			0 & -1
		\end{pmatrix}$ & $\begin{pmatrix}
			0 & 1 \\
			1 & 0
		\end{pmatrix}$ & $\begin{pmatrix}
			0 & -1 \\
			-1& 0
		\end{pmatrix}$ \\ \hline
	\end{tabular}
	\end{center}	
	
	It is easy for readers to verify that they have the same group multiplication table as the symmetry operations. Nevertheless, this is also tedious and boring. Therefore, to realize it, I supply a Python script named \verb!verify_matrix_representation_in_5_1.py! under \verb!../scripts/chapter_05/!.
	
	\end{enumerate}
	
	\end{solution}
	
	% 5.2
	\begin{exercise}

	The table below gives the effects of the transformation operator $\boldsymbol{O}_R$ for the symmetry operation $R$ of the point group $\mathscr{D}_4$ on four functions $f_1$, $f_2$, $f_3$, and $f_4$. Construct a four-dimentional representation of $\mathscr{D}_4$.
	\begin{center}
	\begin{tabular}{c|cccccccc}\hline
	$R =$ & $E$ & $C_4$ & $C^3_4$ & $C_2$ & $C^\prime_{2a}$ & $C^\prime_{2b}$ & $C^{\prime\prime}_{2a}$ & $C^{\prime\prime}_{2b}$ \\ \hline 
	$f_1$ & $f_1$ & $f_2$ & $f_4$ & $f_3$ & $-f_4$ & $-f_2$ & $-f_1$ & $-f_3$ \\
	$f_2$ & $f_2$ & $f_3$ & $f_1$ & $f_4$ & $-f_3$ & $-f_1$ & $-f_4$ & $-f_2$ \\
	$f_3$ & $f_3$ & $f_4$ & $f_2$ & $f_1$ & $-f_2$ & $-f_4$ & $-f_3$ & $-f_1$ \\
	$f_4$ & $f_4$ & $f_1$ & $f_3$ & $f_2$ & $-f_1$ & $-f_3$ & $-f_2$ & $-f_4$ \\ \hline
	\end{tabular}
	\end{center}
		
	\end{exercise}

	\begin{solution}
	
	It is easy to construct a four-dimentional representation of $\mathscr{D}_4$. For example, we find that
	\begin{align*}
		D(C_4)(f_1, f_2, f_3, f_4) &= (D(C_4)f_1, D(C_4)f_2, D(C_4)f_3, D(C_4)f_4) \\
			&= (f_2, f_3, f_4, f_1) \\
			&= (f_1,f_2,f_3,f_4) \begin{pmatrix}
			0 & 0 & 0 & 1 \\
			1 & 0 & 0 & 0 \\
			0 & 1 & 0 & 0 \\
			0 & 0 & 1 & 0 \\
		\end{pmatrix} .
	\end{align*}

	In this way, a four-dimentional representation of $\mathscr{D}_4$ has been constructed in \Tableref{table:matrix_representation_of_d4}.
	\begin{center}
	\captionof{table}{A four-dimentional representation of $\mathscr{D}_4$.}\label{table:matrix_representation_of_d4}
	\begin{tabular}{c|ScScScSc} \hline
		$R$	&	$E$	&	$C_4$	&	$C^\prime_{2a}$	&	$C^\prime_{2b}$	\\ \hline
		$D(R)$	& $\begin{pmatrix}
			1 & 0 & 0 & 0 \\
			0 & 1 & 0 & 0 \\
			0 & 0 & 1 & 0 \\
			0 & 0 & 0 & 1
		\end{pmatrix}$ & $\begin{pmatrix}
			0 & 0 & 0 & 1 \\
			1 & 0 & 0 & 0 \\
			0 & 1 & 0 & 0 \\
			0 & 0 & 1 & 0 \\
		\end{pmatrix}$ & $\begin{pmatrix}
			0 & 0 & 0 & -1 \\
			0 & 0 & -1 & 0 \\
			0 & -1 & 0 & 0 \\
			-1 & 0 & 0 & 0
		\end{pmatrix}$ &
		$\begin{pmatrix}
			0 & -1 & 0 & 0 \\
			-1 & 0 & 0 & 0 \\
			0 & 0 & 0 & -1 \\
			0 & 0 & -1 & 0 
		\end{pmatrix}$ \\ \hline
		$R$	&	$C^3_4$	&	$C_2$	&	$C^{\prime\prime}_{2a}$	&	$C^{\prime\prime}_{2b}$	\\ \hline
		$D(R)$ & $\begin{pmatrix}
			0 & 1 & 0 & 0 \\
			0 & 0 & 1 & 0 \\
			0 & 0 & 0 & 1 \\
			1 & 0 & 0 & 0
		\end{pmatrix}$ & 
		$\begin{pmatrix}
			0 & 0 & 1 & 0 \\
			0 & 0 & 0 & 1 \\
			1 & 0 & 0 & 0 \\
			0 & 1 & 0 & 0
		\end{pmatrix}$ &
		$\begin{pmatrix}
			-1 & 0 & 0 & 0 \\
			0 & 0 & 0 & -1 \\
			0 & 0 & -1 & 0 \\
			0 & -1 & 0 & 0 
		\end{pmatrix}$ &
		$\begin{pmatrix}
			0 & 0 & -1 & 0 \\
			0 & -1 & 0 & 0 \\
			-1 & 0 & 0 & 0 \\
			0 & 0 & 0 & -1 
		\end{pmatrix}$ \\ \hline
	\end{tabular}
	\end{center}

	\end{solution}
	
	\begin{remark}
		In fact, I think that eqn (5-4.2) is hard to remember directly. Compared to eqn (5-2.16), it is easier to remember eqn (5-4.2) by rewriting it as
		\[
			R ( {\bf e}_1 , {\bf e}_2, \ldots, {\bf e}_n ) = ( {\bf e}_1 , {\bf e}_2, \ldots, {\bf e}_n ) D(R).
		\]
	\end{remark}

	% 5.3
	\begin{exercise}
	
	Consider a set of base vectors located on the nuclei of the molecule $\ce{SO2}$ as in the figure below (${\bf e}_3$, ${\bf e}_6$, ${\bf e}_9$  are perpendicular to the page).
	\begin{center}
		\begin{minipage}[t]{0.6\linewidth}
		\centering
		\setlength{\abovecaptionskip}{0.5em}
		\includegraphics[scale=0.85]{../diagrams/chapter_05/SO2.png}
		\end{minipage}\label{fig:SO2}
	\end{center}		
	
	Construct a nine-dimentional matrix representation for the point group to which $\ce{SO2}$ belongs.
	
	\end{exercise}
	
	\begin{solution}
	
	Similar to $\ce{H2O}$, $\ce{SO2}$ belongs to the $\mathscr{C}_{\rm 2v}$. Firstly, we demonstrate the effects of the transformation operators $\boldsymbol{O}_R$ for $\mathscr{C}_{\rm 2v}$, as shown in \Tableref{table:transformation_table_of_so2}.
	
	\begin{center}
	\captionof{table}{Transformation of ${\bf e}_i$ under $\boldsymbol{O}_R$ for the $\mathscr{C}_{\rm 2v}$ point group.}\label{table:transformation_table_of_so2}
	\begin{tabular}{c|cccc}\hline
		$R =$ & 	$E$		& $C_2$ & $\sigma_{xz}$ 	& $\sigma_{yz}$ \\ \hline 
		${\bf e}_1$ &	${\bf e}_1$	& $-{\bf e}_1$	&	$-{\bf e}_1$	&	${\bf e}_1$		\\
		${\bf e}_2$ &	${\bf e}_2$	& ${\bf e}_2$	&	${\bf e}_2$	&	${\bf e}_2$		\\
		${\bf e}_3$ &	${\bf e}_3$	& $-{\bf e}_3$	&	${\bf e}_3$	&	$-{\bf e}_3$		\\
		${\bf e}_4$ &	${\bf e}_4$	& $-{\bf e}_7$	&	$-{\bf e}_7$	&	${\bf e}_4$		\\
		${\bf e}_5$ &	${\bf e}_5$	& ${\bf e}_8$	&	${\bf e}_8$	&	${\bf e}_5$		\\
		${\bf e}_6$ &	${\bf e}_6$	& $-{\bf e}_9$	&	${\bf e}_9$	&	$-{\bf e}_6$		\\
		${\bf e}_7$ &	${\bf e}_7$	& $-{\bf e}_4$	&	$-{\bf e}_4$	&	${\bf e}_7$		\\
		${\bf e}_8$ &	${\bf e}_8$	& ${\bf e}_5$	&	${\bf e}_5$	&	${\bf e}_8$		\\
		${\bf e}_9$ &	${\bf e}_9$	& $-{\bf e}_6$	&	${\bf e}_6$	&	$-{\bf e}_9$		\\ \hline
	\end{tabular}
	\end{center}
	
	Then, we construct the nine-dimentional matrix representation of $\mathscr{C}_{\rm 2v}$, shown in \Tableref{table:matrix_representation_of_so2}.
	\begin{center}
	\captionof{table}{A nine-dimentional matrix representation for the $\mathscr{C}_{\rm 2v}$ point group.}\label{table:matrix_representation_of_so2}
	\begin{tabular}{c|ScSc} \hline
		$R$		&	$E$	&	$C_2$	\\ \hline
		$D(R)$	&	$\begin{pmatrix}
			1	&	0	&	0	&	0	&	0	&	0	&	0	&	0	&	0 \\
			0	&	1	&	0	&	0	&	0	&	0	&	0	&	0	&	0 \\
			0	&	0	&	1	&	0	&	0	&	0	&	0	&	0	&	0 \\
			0	&	0	&	0	&	1	&	0	&	0	&	0	&	0	&	0 \\
			0	&	0	&	0	&	0	&	1	&	0	&	0	&	0	&	0 \\
			0	&	0	&	0	&	0	&	0	&	1	&	0	&	0	&	0 \\
			0	&	0	&	0	&	0	&	0	&	0	&	1	&	0	&	0 \\
			0	&	0	&	0	&	0	&	0	&	0	&	0	&	1	&	0 \\
			0	&	0	&	0	&	0	&	0	&	0	&	0	&	0	&	1
		\end{pmatrix}$ & $\begin{pmatrix}
			-1	&	0	&	0	&	0	&	0	&	0	&	0	&	0	&	0 \\
			0	&	1	&	0	&	0	&	0	&	0	&	0	&	0	&	0 \\
			0	&	0	&	-1	&	0	&	0	&	0	&	0	&	0	&	0 \\
			0	&	0	&	0	&	0	&	0	&	0	&	-1	&	0	&	0 \\
			0	&	0	&	0	&	0	&	0	&	0	&	0	&	1	&	0 \\
			0	&	0	&	0	&	0	&	0	&	0	&	0	&	0	&	-1 \\
			0	&	0	&	0	&	-1	&	0	&	0	&	0	&	0	&	0 \\
			0	&	0	&	0	&	0	&	1	&	0	&	0	&	0	&	0 \\
			0	&	0	&	0	&	0	&	0	&	-1	&	0	&	0	&	0
		\end{pmatrix}$ \\ \hline
		$R$		&	$\sigma_{xz}$	&	$\sigma_{yz}$	\\ \hline
		$D(R)$	&	$\begin{pmatrix}
			-1	&	0	&	0	&	0	&	0	&	0	&	0	&	0	&	0 \\
			0	&	1	&	0	&	0	&	0	&	0	&	0	&	0	&	0 \\
			0	&	0	&	1	&	0	&	0	&	0	&	0	&	0	&	0 \\
			0	&	0	&	0	&	0	&	0	&	0	&	-1	&	0	&	0 \\
			0	&	0	&	0	&	0	&	0	&	0	&	0	&	1	&	0 \\
			0	&	0	&	0	&	0	&	0	&	0	&	0	&	0	&	1 \\
			0	&	0	&	0	&	-1	&	0	&	0	&	0	&	0	&	0 \\
			0	&	0	&	0	&	0	&	1	&	0	&	0	&	0	&	0 \\
			0	&	0	&	0	&	0	&	0	&	1	&	0	&	0	&	0
		\end{pmatrix}$ & $\begin{pmatrix}
			1	&	0	&	0	&	0	&	0	&	0	&	0	&	0	&	0 \\
			0	&	1	&	0	&	0	&	0	&	0	&	0	&	0	&	0 \\
			0	&	0	&	-1	&	0	&	0	&	0	&	0	&	0	&	0 \\
			0	&	0	&	0	&	1	&	0	&	0	&	0	&	0	&	0 \\
			0	&	0	&	0	&	0	&	1	&	0	&	0	&	0	&	0 \\
			0	&	0	&	0	&	0	&	0	&	-1	&	0	&	0	&	0 \\
			0	&	0	&	0	&	0	&	0	&	0	&	1	&	0	&	0 \\
			0	&	0	&	0	&	0	&	0	&	0	&	0	&	1	&	0 \\
			0	&	0	&	0	&	0	&	0	&	0	&	0	&	0	&	-1
		\end{pmatrix}$ \\ \hline
	\end{tabular}
	
	\end{center}	
		
	\end{solution}
	
	% 5.4
	\begin{exercise}
		
	For the point group $\mathscr{D}_{\rm 2h}$:
	\begin{enumerate}[label=(\alph*)]
	
	\item construct a three-dimentional matrix representation using three real p-orbitals as basis functions;
	
	\item construct a five-dimentional matrix representation using five real d-orbitals as basis functions.
	
	\end{enumerate}
		
	\end{exercise}
	
	\begin{solution}
	
	\begin{enumerate}[label=(\alph*)]

	\item It is easy to summarize the transformation of $\orbp_i$ under $\boldsymbol{O}_R$ for the $\mathscr{D}_{\rm 2h}$ point group, demonstrated in \Tableref{table:transformation_table_of_d2h_p_orbitals}.
	\begin{center}
	\captionof{table}{Transformation of $\orbp_i$ under $\boldsymbol{O}_R$ for the $\mathscr{D}_{\rm 2h}$ point group.}\label{table:transformation_table_of_d2h_p_orbitals}
	\begin{tabular}{c|cccccccc}\hline
		$R =$ & 	$E$	& $C_2(z)$ & $C_2(y)$ & $C_2(x)$ & $i$ & $\sigma_{xy}$ & $\sigma_{xz}$ & $\sigma_{yz}$ \\ \hline 
		$\orbp_1 \equiv \orbp_{x}$ &	$\orbp_1$ & 	$-\orbp_1$ & $-\orbp_1$ & $\orbp_1$ & $-\orbp_1$ & $\orbp_1$ & $\orbp_1$ & $-\orbp_1$  \\
		$\orbp_2 \equiv \orbp_{y}$ &	$\orbp_2$ & 	$-\orbp_2$ & $\orbp_2$ & $-\orbp_2$ & $-\orbp_2$ & $\orbp_2$ & $-\orbp_2$ & $\orbp_2$	\\
		$\orbp_3 \equiv \orbp_{z}$ &	$\orbp_3$ & 	$\orbp_3$ & $-\orbp_3$ & $-\orbp_3$ & $-\orbp_3$ & $-\orbp_3$ & $\orbp_3$ & $\orbp_3$	\\ \hline
	\end{tabular}
	\end{center}
	
	Then, a three-dimentional matrix representation for the $\mathscr{D}_{\rm 2h}$ point group can be constructed in the same way as exercise 5.2.
	\begin{center}
	\captionof{table}{A three-dimentional matrix representation for the $\mathscr{D}_{\rm 2h}$ point group.}\label{table:matrix_representation_of_d2h_p_orbitals}
	\begin{tabular}{c|ScScScSc}\hline
		$R$ & $E$ & $C_2(z)$ & $C_2(y)$ & $C_2(x)$ \\ \hline
		$D(R)$& $\begin{pmatrix}
			1 & 0 & 0 \\
			0 & 1 & 0 \\
			0 & 0 & 1
		\end{pmatrix}$ & $\begin{pmatrix}
			-1 & 0 & 0 \\
			0 & -1 & 0 \\
			0 & 0 & 1
		\end{pmatrix}$ & $\begin{pmatrix}
			-1 & 0 & 0 \\
			0 & 1 & 0 \\
			0 & 0 & -1
		\end{pmatrix}$ & $\begin{pmatrix}
			1 & 0 & 0 \\
			0 & -1 & 0 \\
			0 & 0 & -1
		\end{pmatrix}$ \\ \hline
		$R$ & $i$ & $\sigma(xy)$ & $\sigma(xz)$ & $\sigma(yz)$ \\ \hline
		$D(R)$ & $\begin{pmatrix}
			-1 & 0 & 0 \\
			0 & -1 & 0 \\
			0 & 0 & -1
		\end{pmatrix}$ & $\begin{pmatrix}
			1 & 0 & 0 \\
			0 & 1 & 0 \\
			0 & 0 & -1
		\end{pmatrix}$ & $\begin{pmatrix}
			1 & 0 & 0 \\
			0 & -1 & 0 \\
			0 & 0 & 1
		\end{pmatrix}$ & $\begin{pmatrix}
			-1 & 0 & 0 \\
			0 & 1 & 0 \\
			0 & 0 & 1
		\end{pmatrix}$ \\ \hline
	\end{tabular}
	\end{center}
	
	By the way, taking the 2p orbitals as an example, the shapes of the three p orbitals can be seen in \Figref{fig:p_orbitals}.
	\begin{center}
	\begin{tabular}{ccc}
		\begin{minipage}[t]{0.3\linewidth}
		\centering
		\setlength{\abovecaptionskip}{-2em}
		\includegraphics[scale=0.28]{../diagrams/chapter_05/orbital_p_1.png}
		\captionof*{figure}{$\orbp_1$}
		\end{minipage} & 
		\begin{minipage}[t]{0.3\linewidth}
		\centering
		\setlength{\abovecaptionskip}{-2em}
		\includegraphics[scale=0.28]{../diagrams/chapter_05/orbital_p_2.png}
		\captionof*{figure}{$\orbp_2$}
		\end{minipage} & 
		\begin{minipage}[t]{0.3\linewidth}
		\centering
		\setlength{\abovecaptionskip}{-2em}
		\includegraphics[scale=0.28]{../diagrams/chapter_05/orbital_p_3.png}
		\captionof*{figure}{$\orbp_3$}
		\end{minipage}
	\end{tabular}
	\captionof{figure}{Diagrams of three 2p-orbitals.}\label{fig:p_orbitals}
	\end{center}
	
	\item It is also easy to summarize the transformation of $\orbd_i$ under $\boldsymbol{O}_R$ for the $\mathscr{D}_{\rm 2h}$ point group, demonstrated in \Tableref{table:transformation_table_of_d2h_d_orbitals}.
	\begin{center}
	\captionof{table}{Transformation of $\orbd_i$ under $\boldsymbol{O}_R$ for the $\mathscr{D}_{\rm 2h}$ point group.}\label{table:transformation_table_of_d2h_d_orbitals}
	\begin{tabular}{c|cccccccc}\hline
		$R =$ & 	$E$	& $C_2(z)$ & $C_2(y)$ & $C_2(x)$ & $i$ & $\sigma_{xy}$ & $\sigma_{xz}$ & $\sigma_{yz}$ \\ \hline 
		$\orbd_1 \equiv \orbd_{x^2 - y^2}$ &	$\orbd_1$ & 	$\orbd_1$ & $\orbd_1$ & $\orbd_1$ & $\orbd_1$ & $\orbd_1$ & $\orbd_1$ & $\orbd_1$ \\
		$\orbd_2 \equiv \orbd_{xy}$ & $\orbd_2$ & 	$\orbd_2$ & $-\orbd_2$ & $-\orbd_2$ & $\orbd_2$ & $\orbd_2$ & $-\orbd_2$ & $-\orbd_2$		\\
		$\orbd_3 \equiv \orbd_{xz}$ & $\orbd_3$ & 	$-\orbd_3$ & $\orbd_3$ & $-\orbd_3$ & $\orbd_3$ & $-\orbd_3$ & $\orbd_3$ & $-\orbd_3$		\\
		$\orbd_4 \equiv \orbd_{yz}$ & $\orbd_4$ & 	$-\orbd_4$ & $-\orbd_4$ & $\orbd_4$ & $\orbd_4$ & $-\orbd_4$ & $-\orbd_4$ & $\orbd_4$			\\
		$\orbd_5 \equiv \orbd_{z^2}$ & $\orbd_5$ & 	$\orbd_5$ & $\orbd_5$ & $\orbd_5$ & $\orbd_5$ & $\orbd_5$ & $\orbd_5$ & $\orbd_5$	\\ \hline
	\end{tabular}
	\end{center}	
	
	Then, a five-dimentional matrix representation for the $\mathscr{D}_{\rm 2h}$ point group can be constructed.
	\begin{center}
	\captionof{table}{A five-dimentional matrix representation for the $\mathscr{D}_{\rm 2h}$ point group.}\label{table:matrix_representation_of_d2h_d_orbitals}
	\begin{tabular}{c|ScScSc}\hline
		$R$ & $E$ & $C_2(z)$ & $C_2(y)$ \\ \hline
		$D(R)$& $\begin{pmatrix}
			1 & 0 & 0 & 0 & 0 \\
			0 & 1 & 0 & 0 & 0 \\
			0 & 0 & 1 & 0 & 0 \\
			0 & 0 & 0 & 1 & 0 \\
			0 & 0 & 0 & 0 & 1
		\end{pmatrix}$ & $\begin{pmatrix}
			1 & 0 & 0 & 0 & 0 \\
			0 & 1 & 0 & 0 & 0 \\
			0 & 0 & -1 & 0 & 0 \\
			0 & 0 & 0 & -1 & 0 \\
			0 & 0 & 0 & 0 & 1
		\end{pmatrix}$ & $\begin{pmatrix}
			1 & 0 & 0 & 0 & 0 \\
			0 & -1 & 0 & 0 & 0 \\
			0 & 0 & 1 & 0 & 0 \\
			0 & 0 & 0 & -1 & 0 \\
			0 & 0 & 0 & 0 & 1
		\end{pmatrix}$ \\ \hline 
		$R$ & $C_2(x)$ &  $i$ & $\sigma(xy)$ \\ \hline
		$D(R)$ & $\begin{pmatrix}
			1 & 0 & 0 & 0 & 0 \\
			0 & -1 & 0 & 0 & 0 \\
			0 & 0 & -1 & 0 & 0 \\
			0 & 0 & 0 & 1 & 0 \\
			0 & 0 & 0 & 0 & 1
		\end{pmatrix}$ & $\begin{pmatrix}
			1 & 0 & 0 & 0 & 0 \\
			0 & 1 & 0 & 0 & 0 \\
			0 & 0 & 1 & 0 & 0 \\
			0 & 0 & 0 & 1 & 0 \\
			0 & 0 & 0 & 0 & 1
		\end{pmatrix}$ & $\begin{pmatrix}
			1 & 0 & 0 & 0 & 0 \\
			0 & 1 & 0 & 0 & 0 \\
			0 & 0 & -1 & 0 & 0 \\
			0 & 0 & 0 & -1 & 0 \\
			0 & 0 & 0 & 0 & 1
		\end{pmatrix}$ \\ \hline
		$R$ & $\sigma(xz)$ & $\sigma(yz)$ & \\ \hline
		$D(R)$ & $\begin{pmatrix}
			1 & 0 & 0 & 0 & 0 \\
			0 & -1 & 0 & 0 & 0 \\
			0 & 0 & 1 & 0 & 0 \\
			0 & 0 & 0 & -1 & 0 \\
			0 & 0 & 0 & 0 & 1
		\end{pmatrix}$ & $\begin{pmatrix}
			1 & 0 & 0 & 0 & 0 \\
			0 & -1 & 0 & 0 & 0 \\
			0 & 0 & -1 & 0 & 0 \\
			0 & 0 & 0 & 1 & 0 \\
			0 & 0 & 0 & 0 & 1
		\end{pmatrix}$ \\ \hline
	\end{tabular}
	\end{center}
	
	Finally, taking the 3d orbitals as an example, their shapes can be seen in \Figref{fig:d_orbitals}.
	
	\begin{center}
	\begin{tabular}{ccc}
		\begin{minipage}[t]{0.3\linewidth}
		\centering
		\setlength{\abovecaptionskip}{-2em}
		\includegraphics[scale=0.26]{../diagrams/chapter_05/orbital_d_1.png}
		\captionof*{figure}{$\orbd_1$}
		\end{minipage} & 
		\begin{minipage}[t]{0.3\linewidth}
		\centering
		\setlength{\abovecaptionskip}{-2em}
		\includegraphics[scale=0.26]{../diagrams/chapter_05/orbital_d_2.png}
		\captionof*{figure}{$\orbd_2$}
		\end{minipage} & 
		\begin{minipage}[t]{0.3\linewidth}
		\centering
		\setlength{\abovecaptionskip}{-2em}
		\includegraphics[scale=0.26]{../diagrams/chapter_05/orbital_d_3.png}
		\captionof*{figure}{$\orbd_3$}
		\end{minipage} \\
		\begin{minipage}[t]{0.3\linewidth}
		\centering
		\setlength{\abovecaptionskip}{-2em}
		\includegraphics[scale=0.27]{../diagrams/chapter_05/orbital_d_4.png}
		\captionof*{figure}{$\orbd_4$}
		\end{minipage} & 
		\begin{minipage}[t]{0.3\linewidth}
		\centering
		\setlength{\abovecaptionskip}{-2em}
		\includegraphics[scale=0.27]{../diagrams/chapter_05/orbital_d_5.png}
		\captionof*{figure}{$\orbd_5$}
		\end{minipage} 
	\end{tabular}
	\captionof{figure}{Diagrams of three 3d-orbitals.}\label{fig:d_orbitals}
	\end{center}
	
	\end{enumerate}
	
	\end{solution}
	
	\begin{remark}

	In this exercise, it is evident that for molecules belonging to a point group containing an $i$, if a certain symmetry operation $A=Bi=iB$, then $A$ and $B$ share the same matrix representation in a basis including only gerade orbitals due to $D(i)=I_n$ under this basis, as seen in part (b). Conversely, for a basis of only ungerade orbitals, $D(i)=-I_n$, which implies $D(A)=-D(B)$, as in part (a).
	
	\end{remark}
	
	% 5.5
	\begin{exercise}
	
	Consider the planar trivinylmethyl radical with seven $\pi$-orbitals located as shown below:
	
	\begin{center}
		\begin{minipage}[t]{0.6\linewidth}
		\centering
		\setlength{\abovecaptionskip}{0.5em}
		\includegraphics[scale=1.0]{../diagrams/chapter_05/trivinylmethyl_radical.png}
		\end{minipage}\label{fig:trivinylmethyl_radical}
	\end{center}	
	
	Using these $\pi$-orbitals as basis functions, construct a seven-dimensional representation of the $\mathscr{C}_3$ point group.
	
	\end{exercise}
	
	\begin{solution}
	
	We summarize the transformation of $\orbp_i$ under $\boldsymbol{O}_R$ for the $\mathscr{C}_3$ point group, demonstrated in \Tableref{table:transformation_table_of_c3}.
	\begin{center}
	\captionof{table}{Transformation of $\orbp_i$ under $\boldsymbol{O}_R$ for the $\mathscr{C}_3$ point group.}\label{table:transformation_table_of_c3}
	\begin{tabular}{c|ccc} \hline
		$R = $ 		&		$E$		&	$C_3$		&	$C^2_3$		\\ \hline
		$\orbp_1$	&	$\orbp_1$	&	$\orbp_2$	&	$\orbp_3$	\\
		$\orbp_2$	&	$\orbp_2$	&	$\orbp_3$	&	$\orbp_1$	\\
		$\orbp_3$	&	$\orbp_3$	&	$\orbp_1$	&	$\orbp_2$	\\
		$\orbp_4$	&	$\orbp_4$	&	$\orbp_5$	&	$\orbp_6$	\\
		$\orbp_5$	&	$\orbp_5$	&	$\orbp_6$	&	$\orbp_4$	\\
		$\orbp_6$	&	$\orbp_6$	&	$\orbp_4$	&	$\orbp_5$	\\
		$\orbp_7$	&	$\orbp_7$	&	$\orbp_7$	&	$\orbp_7$	\\ \hline
	\end{tabular}
	\end{center}
	
	Then, a seven-dimentional matrix representation for the $\mathscr{C}_3$ point group can be constructed.
	\begin{center}
	\captionof{table}{A seven-dimentional matrix representation for the $\mathscr{C}_3$ point group.}\label{table:matrix_representation_of_c3}
	\begin{tabular}{c|ScScSc} \hline
		$R$		&	$E$	&	$C_3$		&	$C^2_3$		\\ \hline
		$D(R)$	& $\begin{pmatrix}
			1 & 0 & 0 & 0 & 0 & 0 & 0 \\
			0 & 1 & 0 & 0 & 0 & 0 & 0 \\
			0 & 0 & 1 & 0 & 0 & 0 & 0 \\
			0 & 0 & 0 & 1 & 0 & 0 & 0 \\
			0 & 0 & 0 & 0 & 1 & 0 & 0 \\
			0 & 0 & 0 & 0 & 0 & 1 & 0 \\
			0 & 0 & 0 & 0 & 0 & 0 & 1
		\end{pmatrix}$ & $\begin{pmatrix}
			0 & 0 & 1 & 0 & 0 & 0 & 0 \\
			1 & 0 & 0 & 0 & 0 & 0 & 0 \\
			0 & 1 & 0 & 0 & 0 & 0 & 0 \\
			0 & 0 & 0 & 0 & 0 & 1 & 0 \\
			0 & 0 & 0 & 1 & 0 & 0 & 0 \\
			0 & 0 & 0 & 0 & 1 & 0 & 0 \\
			0 & 0 & 0 & 0 & 0 & 0 & 1
		\end{pmatrix}$ & $\begin{pmatrix}
			0 & 1 & 0 & 0 & 0 & 0 & 0 \\
			0 & 0 & 1 & 0 & 0 & 0 & 0 \\
			1 & 0 & 0 & 0 & 0 & 0 & 0 \\
			0 & 0 & 0 & 0 & 1 & 0 & 0 \\
			0 & 0 & 0 & 0 & 0 & 1 & 0 \\
			0 & 0 & 0 & 1 & 0 & 0 & 0 \\
			0 & 0 & 0 & 0 & 0 & 0 & 1
		\end{pmatrix}$ \\ \hline
	\end{tabular}
	\end{center}
	
	\end{solution}	

\end{document}