\documentclass[a4paper]{book}

\usepackage{afterpage}
\usepackage[hypcap=false]{caption}
\usepackage{enumitem}	% 定制enumerate标号
\usepackage{geometry}
\geometry{%
	left=2cm,%
	right=2cm,%
	top=2cm,%
	bottom=2cm,%
	bindingoffset=0cm
}
\usepackage{hyperref}
\hypersetup{
    colorlinks=true,            %链接颜色
    linkcolor=blue,             %内部链接
    filecolor=magenta,          %本地文档
    urlcolor=cyan,              %网址链接
    pdftitle={Overleaf Example},
    pdfpagemode=FullScreen,
}
\usepackage[none]{hyphenat}	% 阻止长单词分在两行
\usepackage{longtable}
\usepackage{mathrsfs}	% 提供\mathscr字体
\usepackage[version=4]{mhchem}
\usepackage{multirow}
\usepackage{subcaption}

\RequirePackage[many]{tcolorbox}
\tcbset{
    boxed title style={colback=magenta},
	breakable,
	enhanced,
	sharp corners,
	attach boxed title to top left={yshift=-\tcboxedtitleheight,  yshifttext=-.75\baselineskip},
	boxed title style={boxsep=1pt,sharp corners},
    fonttitle=\bfseries\sffamily,
}

\definecolor{skyblue}{rgb}{0.54, 0.81, 0.94}

\newtcolorbox[auto counter, number within=chapter, number format=\arabic]{exercise}[1][]{
    title={Exercise~\thetcbcounter},
    colframe=skyblue,
    colback=skyblue!12!white,
    boxed title style={colback=skyblue},
    overlay unbroken and first={
        \node[below right,font=\small,color=skyblue,text width=.8\linewidth]
        at (title.north east) {#1};
    }
}

\newtcolorbox[auto counter, number within=chapter, number format=\arabic]{solution}[1][]{
%    top=2ex,
%    boxrule=0pt,
%    leftrule=1.4pt,
    title={Solution~\thetcbcounter},
    colframe=teal!60!green,
    colback=green!12!white,
    boxed title style={colback=teal!60!green},
    overlay unbroken and first={
        \node[below right,font=\small,color=red,text width=.8\linewidth]
        at (title.north east) {#1};
    }
}

\newcommand{\AO}{{\rm AO}}
\newcommand{\Heff}{H^{\rm eff,\pi}}
\newcommand{\Hp}{H^\prime}
\newcommand{\Sp}{S^\prime}
\newcommand{\RRR}{{\rm R}^3}
\newcommand\Figref[1]{Fig \ref{#1}}
\newcommand\Tableref[1]{Table \ref{#1}}
\newcommand{\orb}[1]{{\rm #1}}
\newcommand{\orbp}{\orb{p}}

\allowdisplaybreaks

\begin{document}

	\setcounter{chapter}{5}
	
	% 5.1
	\begin{exercise}
	
	Consider the following planar symmetric figure.
	
	\begin{center}
		\begin{minipage}[t]{0.6\linewidth}
		\centering
		\setlength{\abovecaptionskip}{0.5em}
		\includegraphics[scale=1.0]{../diagrams/chapter_05/two_triangles.png}
		\end{minipage}\label{fig:two_triangles}
	\end{center}	
	
	\begin{enumerate}[label=(\alph*)]
	
	\item Determine the distinct symmetry operations which take it into itself; construct the group multiplication table for these operations, and identify the point group to which this figure belongs.
	
	\item Find a set of two-dimensional matrices which are in one-to-one correspondence with the above symmetry operations, and verify that they have the same group multiplication table as the symmetry operations.
	
	\end{enumerate}
	
	\end{exercise}

	\begin{solution}
	
	\begin{enumerate}[label=(\alph*)]
	
	\item To identify the symmetry operations, we look for transformations that leave the figure indistinguishable from its original state:
	
		\begin{itemize}
	
		\item $E$ (Identity): The "do nothing" operation.
		
		\item $C_2(z)$: A 180° rotation about the $z$-axis (the axis passing through the center perpendicular to the screen).
		
		\item $\sigma_v(xz)$: A mirror plane passing through the horizontal axis of the figure.
		
		\item $\sigma_v(yz)$: A mirror plane passing through the vertical center point, bisecting the "bow-tie."
	
		\end{itemize}
			
	Point Group: Because it has a $C_2$ axis and two vertical mirror planes ($\sigma_v$), the point group is $C_{2v}$.
	
	\item 
	
	\begin{equation}
		E \rightarrow \begin{pmatrix}
			1 & 0 \\
			0 & 1
		\end{pmatrix} , 			
	\end{equation}		
	
	\end{enumerate}
	
	
	
	\end{solution}
	
	% 5.2
	\begin{exercise}

	The table below gives the effects of the transformation operator $O_R$ for the symmetry operation $R$ of the point group $\mathscr{D}_4$ on four functions $f_1$, $f_2$, $f_3$, and $f_4$. Construct a four-dimentional representation of $\mathscr{D}_4$.
	\begin{center}
	\begin{tabular}{c|cccccccc}\hline
	$R =$ & $E$ & $C_4$ & $C^3_4$ & $C_2$ & $C^\prime_{2a}$ & $C^\prime_{2b}$ & $C^{\prime\prime}_{2a}$ & $C^{\prime\prime}_{2b}$ \\ \hline 
	$f_1$ & $f_1$ & $f_2$ & $f_4$ & $f_3$ & $-f_4$ & $-f_2$ & $-f_1$ & $-f_3$ \\
	$f_2$ & $f_2$ & $f_3$ & $f_1$ & $f_4$ & $-f_3$ & $-f_1$ & $-f_4$ & $-f_2$ \\
	$f_3$ & $f_3$ & $f_4$ & $f_2$ & $f_1$ & $-f_2$ & $-f_4$ & $-f_3$ & $-f_1$ \\
	$f_4$ & $f_4$ & $f_1$ & $f_3$ & $f_2$ & $-f_1$ & $-f_3$ & $-f_2$ & $-f_4$ \\ \hline
	\end{tabular}
	\end{center}
		
	\end{exercise}

	\begin{solution}
	
	\begin{align}
		&D(E) = \begin{pmatrix}
			1 & 0 & 0 & 0 \\
			0 & 1 & 0 & 0 \\
			0 & 0 & 1 & 0 \\
			0 & 0 & 0 & 1
		\end{pmatrix} , 
		&D(C_4) = \begin{pmatrix}
			0 & 1 & 0 & 0 \\
			0 & 0 & 1 & 0 \\
			0 & 0 & 0 & 1 \\
			1 & 0 & 0 & 0 \\
		\end{pmatrix} , \\
		&D(C^3_4) = \begin{pmatrix}
			0 & 0 & 0 & 1 \\
			1 & 0 & 0 & 0 \\
			0 & 1 & 0 & 0 \\
			0 & 0 & 1 & 0
		\end{pmatrix} ,
		&D(C_2) = \begin{pmatrix}
			0 & 0 & 1 & 0 \\
			0 & 0 & 0 & 1 \\
			1 & 0 & 0 & 0 \\
			0 & 1 & 0 & 0
		\end{pmatrix} , \\
		&D(C^\prime_{2a}) = \begin{pmatrix}
			0 & 0 & 0 & -1 \\
			0 & 0 & -1 & 0 \\
			0 & -1 & 0 & 0 \\
			-1 & 0 & 0 & 0
		\end{pmatrix} ,
		&D(C^\prime_{2b}) = \begin{pmatrix}
			0 & -1 & 0 & 0 \\
			-1 & 0 & 0 & 0 \\
			0 & 0 & 0 & -1 \\
			0 & 0 & -1 & 0 
		\end{pmatrix} , \\
		&D(C^{\prime\prime}_{2a}) = \begin{pmatrix}
			-1 & 0 & 0 & 0 \\
			0 & 0 & 0 & -1 \\
			0 & 0 & -1 & 0 \\
			0 & -1 & 0 & 0 
		\end{pmatrix} ,
		&D(C^{\prime\prime}_{2b}) = \begin{pmatrix}
			0 & 0 & -1 & 0 \\
			0 & -1 & 0 & 0 \\
			-1 & 0 & 0 & 0 \\
			0 & 0 & 0 & -1 
		\end{pmatrix} ,
	\end{align}

	\end{solution}

	% 5.3
	\begin{exercise}
	
	Consider a set of base vectors located on the nuclei of the molecule $\ce{SO2}$ as in the figure below (${\bf e}_3$, ${\bf e}_6$, ${\bf e}_9$  are perpendicular to the page).
	\begin{center}
		\begin{minipage}[t]{0.6\linewidth}
		\centering
		\setlength{\abovecaptionskip}{0.5em}
		\includegraphics[scale=1.0]{../diagrams/chapter_05/SO2.png}
		\end{minipage}\label{fig:SO2}
	\end{center}		
	
	Construct a nine-dimentional matrix representation for the point group to which $\ce{SO2}$ belongs.
	
	\end{exercise}
	
	\begin{solution}
	
	\begin{center}
	\begin{tabular}{ccc}
		\begin{minipage}[t]{0.3\linewidth}
		\centering
		\setlength{\abovecaptionskip}{0.5em}
		\includegraphics[scale=0.2]{../diagrams/chapter_05/SO2-C2.png}
		\captionof*{figure}{$C_2$}
		\end{minipage} & 
		\begin{minipage}[t]{0.3\linewidth}
		\centering
		\setlength{\abovecaptionskip}{0.5em}
		\includegraphics[scale=0.2]{../diagrams/chapter_05/SO2-sigma_xz.png}
		\captionof*{figure}{$\sigma_{\rm v}(xz)$}
		\end{minipage} & 
		\begin{minipage}[t]{0.3\linewidth}
		\centering
		\setlength{\abovecaptionskip}{0.5em}
		\includegraphics[scale=0.2]{../diagrams/chapter_05/SO2-sigma_yz.png}
		\captionof*{figure}{$\sigma_{\rm v}(yz)$}
		\end{minipage} 
	\end{tabular}
	\captionof{figure}{All symmetry elements of $\ce{SO2}$ except the identity $E$.}\label{fig:SO2_symmetry_elements}
	\end{center}
	
	\begin{center}
	\begin{tabular}{c|cccc}\hline
		$R =$ & 	$E$		& $C_2$ & $\sigma_{xz}$ 	& $\sigma_{yz}$ \\ \hline 
		$e_1$ &	$e_1$	& $-e_1$	&	$-e_1$	&	$e_1$		\\
		$e_2$ &	$e_2$	& $e_2$	&	$e_2$	&	$e_2$		\\
		$e_3$ &	$e_3$	& $-e_3$	&	$e_3$	&	$-e_3$		\\
		$e_4$ &	$e_4$	& $-e_7$	&	$-e_7$	&	$e_4$		\\
		$e_5$ &	$e_5$	& $e_8$	&	$e_8$	&	$e_5$		\\
		$e_6$ &	$e_6$	& $-e_9$	&	$e_9$	&	$-e_6$		\\
		$e_7$ &	$e_7$	& $-e_4$	&	$-e_4$	&	$e_7$		\\
		$e_8$ &	$e_8$	& $e_5$	&	$e_5$	&	$e_8$		\\
		$e_9$ &	$e_9$	& $-e_6$	&	$e_6$	&	$-e_9$		\\ \hline
	\end{tabular}
	\end{center}
		
	\begin{align}
		D(E) &= \begin{pmatrix}
			1	&	0	&	0	&	0	&	0	&	0	&	0	&	0	&	0 \\
			0	&	1	&	0	&	0	&	0	&	0	&	0	&	0	&	0 \\
			0	&	0	&	1	&	0	&	0	&	0	&	0	&	0	&	0 \\
			0	&	0	&	0	&	1	&	0	&	0	&	0	&	0	&	0 \\
			0	&	0	&	0	&	0	&	1	&	0	&	0	&	0	&	0 \\
			0	&	0	&	0	&	0	&	0	&	1	&	0	&	0	&	0 \\
			0	&	0	&	0	&	0	&	0	&	0	&	1	&	0	&	0 \\
			0	&	0	&	0	&	0	&	0	&	0	&	0	&	1	&	0 \\
			0	&	0	&	0	&	0	&	0	&	0	&	0	&	0	&	1
		\end{pmatrix} , \\
		D(C_2) &= \begin{pmatrix}
			-1	&	0	&	0	&	0	&	0	&	0	&	0	&	0	&	0 \\
			0	&	1	&	0	&	0	&	0	&	0	&	0	&	0	&	0 \\
			0	&	0	&	-1	&	0	&	0	&	0	&	0	&	0	&	0 \\
			0	&	0	&	0	&	0	&	0	&	0	&	-1	&	0	&	0 \\
			0	&	0	&	0	&	0	&	0	&	0	&	0	&	1	&	0 \\
			0	&	0	&	0	&	0	&	0	&	0	&	0	&	0	&	-1 \\
			0	&	0	&	0	&	-1	&	0	&	0	&	0	&	0	&	0 \\
			0	&	0	&	0	&	0	&	1	&	0	&	0	&	0	&	0 \\
			0	&	0	&	0	&	0	&	0	&	-1	&	0	&	0	&	0
		\end{pmatrix} , \\
		D(\sigma_{xz}) &= \begin{pmatrix}
			-1	&	0	&	0	&	0	&	0	&	0	&	0	&	0	&	0 \\
			0	&	1	&	0	&	0	&	0	&	0	&	0	&	0	&	0 \\
			0	&	0	&	1	&	0	&	0	&	0	&	0	&	0	&	0 \\
			0	&	0	&	0	&	0	&	0	&	0	&	-1	&	0	&	0 \\
			0	&	0	&	0	&	0	&	0	&	0	&	0	&	1	&	0 \\
			0	&	0	&	0	&	0	&	0	&	0	&	0	&	0	&	1 \\
			0	&	0	&	0	&	-1	&	0	&	0	&	0	&	0	&	0 \\
			0	&	0	&	0	&	0	&	1	&	0	&	0	&	0	&	0 \\
			0	&	0	&	0	&	0	&	0	&	1	&	0	&	0	&	0
		\end{pmatrix} , \\
		D(\sigma_{yz}) &= \begin{pmatrix}
			1	&	0	&	0	&	0	&	0	&	0	&	0	&	0	&	0 \\
			0	&	1	&	0	&	0	&	0	&	0	&	0	&	0	&	0 \\
			0	&	0	&	-1	&	0	&	0	&	0	&	0	&	0	&	0 \\
			0	&	0	&	0	&	1	&	0	&	0	&	0	&	0	&	0 \\
			0	&	0	&	0	&	0	&	1	&	0	&	0	&	0	&	0 \\
			0	&	0	&	0	&	0	&	0	&	-1	&	0	&	0	&	0 \\
			0	&	0	&	0	&	0	&	0	&	0	&	1	&	0	&	0 \\
			0	&	0	&	0	&	0	&	0	&	0	&	0	&	1	&	0 \\
			0	&	0	&	0	&	0	&	0	&	0	&	0	&	0	&	-1
		\end{pmatrix} .
	\end{align}			
		
	\end{solution}
	
	% 5.4
	\begin{exercise}
		
	For the point group $\mathscr{D}_{\rm 2h}$:
	\begin{enumerate}[label=(\alph*)]
	
	\item construct a three-dimentional matrix representation using three real p-orbitals as basis functions;
	
	\item construct a five-dimentional matrix representation using five real d-orbitals as basis functions.
	
	\end{enumerate}
		
	\end{exercise}
	
	\begin{solution}
	
	\begin{enumerate}[label=(\alph*)]

	\item 
	
	\begin{align}
		&D(E) = \begin{pmatrix}
			1 & 0 & 0 \\
			0 & 1 & 0 \\
			0 & 0 & 1
		\end{pmatrix},
		&D(C_2(z)) = \begin{pmatrix}
			-1 & 0 & 0 \\
			0 & -1 & 0 \\
			0 & 0 & 1
		\end{pmatrix}, \\
		&D(C_2(y)) = \begin{pmatrix}
			-1 & 0 & 0 \\
			0 & 1 & 0 \\
			0 & 0 & -1
		\end{pmatrix}, 
		&D(C_2(x)) = \begin{pmatrix}
			1 & 0 & 0 \\
			0 & -1 & 0 \\
			0 & 0 & -1
		\end{pmatrix}, \\
		&D(i) = \begin{pmatrix}
			-1 & 0 & 0 \\
			0 & -1 & 0 \\
			0 & 0 & -1
		\end{pmatrix}, 
		&D(\sigma(xy)) = \begin{pmatrix}
			1 & 0 & 0 \\
			0 & 1 & 0 \\
			0 & 0 & -1
		\end{pmatrix}, \\
		&D(\sigma(xz)) = \begin{pmatrix}
			1 & 0 & 0 \\
			0 & -1 & 0 \\
			0 & 0 & 1
		\end{pmatrix},
		&D(\sigma(yz)) = \begin{pmatrix}
			-1 & 0 & 0 \\
			0 & 1 & 0 \\
			0 & 0 & 1
		\end{pmatrix}.
	\end{align}
	
	\item 
	
	\begin{align}
		&D(E) = \begin{pmatrix}
			1 & 0 & 0 & 0 & 0 \\
			0 & 1 & 0 & 0 & 0 \\
			0 & 0 & 1 & 0 & 0 \\
			0 & 0 & 0 & 1 & 0 \\
			0 & 0 & 0 & 0 & 1
		\end{pmatrix}, 
		&D(C_2(z)) = \begin{pmatrix}
			1 & 0 & 0 & 0 & 0 \\
			0 & 1 & 0 & 0 & 0 \\
			0 & 0 & -1 & 0 & 0 \\
			0 & 0 & 0 & -1 & 0 \\
			0 & 0 & 0 & 0 & 1
		\end{pmatrix}, \\
		&D(C_2(y)) = \begin{pmatrix}
			1 & 0 & 0 & 0 & 0 \\
			0 & -1 & 0 & 0 & 0 \\
			0 & 0 & 1 & 0 & 0 \\
			0 & 0 & 0 & -1 & 0 \\
			0 & 0 & 0 & 0 & 1
		\end{pmatrix}, 
		&D(C_2(x)) = \begin{pmatrix}
			1 & 0 & 0 & 0 & 0 \\
			0 & -1 & 0 & 0 & 0 \\
			0 & 0 & -1 & 0 & 0 \\
			0 & 0 & 0 & 1 & 0 \\
			0 & 0 & 0 & 0 & 1
		\end{pmatrix}, \\
		&D(i) = \begin{pmatrix}
			1 & 0 & 0 & 0 & 0 \\
			0 & 1 & 0 & 0 & 0 \\
			0 & 0 & 1 & 0 & 0 \\
			0 & 0 & 0 & 1 & 0 \\
			0 & 0 & 0 & 0 & 1
		\end{pmatrix}, 
		&D(\sigma(xy)) = \begin{pmatrix}
			1 & 0 & 0 & 0 & 0 \\
			0 & 1 & 0 & 0 & 0 \\
			0 & 0 & -1 & 0 & 0 \\
			0 & 0 & 0 & -1 & 0 \\
			0 & 0 & 0 & 0 & 1
		\end{pmatrix}, \\
		&D(\sigma(xz)) = \begin{pmatrix}
			1 & 0 & 0 & 0 & 0 \\
			0 & -1 & 0 & 0 & 0 \\
			0 & 0 & 1 & 0 & 0 \\
			0 & 0 & 0 & -1 & 0 \\
			0 & 0 & 0 & 0 & 1
		\end{pmatrix},
		&D(\sigma(yz)) = \begin{pmatrix}
			1 & 0 & 0 & 0 & 0 \\
			0 & -1 & 0 & 0 & 0 \\
			0 & 0 & -1 & 0 & 0 \\
			0 & 0 & 0 & 1 & 0 \\
			0 & 0 & 0 & 0 & 1
		\end{pmatrix},
	\end{align}
	
	\end{enumerate}
	
	\end{solution}
	
	% 5.5
	\begin{exercise}
	
	Consider the planar trivinylmethyl radical with seven $\pi$-orbitals located as shown below:
	
	\begin{center}
		\begin{minipage}[t]{0.6\linewidth}
		\centering
		\setlength{\abovecaptionskip}{0.5em}
		\includegraphics[scale=1.0]{../diagrams/chapter_05/trivinylmethyl_radical.png}
		\end{minipage}\label{fig:trivinylmethyl_radical}
	\end{center}	
	
	Using these $\pi$-orbitals as basis functions, construct a seven-dimensional representation of the $\mathscr{C}_3$ point group.
	
	\end{exercise}
	
	\begin{solution}
	
	\begin{center}
	\begin{tabular}{c|ccc} \hline
		$R = $ 		&		$E$		&	$C_3$		&	$C^2_3$		\\ \hline
		$\orbp_1$	&	$\orbp_1$	&	$\orbp_2$	&	$\orbp_3$	\\
		$\orbp_2$	&	$\orbp_2$	&	$\orbp_3$	&	$\orbp_1$	\\
		$\orbp_3$	&	$\orbp_3$	&	$\orbp_1$	&	$\orbp_2$	\\
		$\orbp_4$	&	$\orbp_4$	&	$\orbp_5$	&	$\orbp_6$	\\
		$\orbp_5$	&	$\orbp_5$	&	$\orbp_6$	&	$\orbp_4$	\\
		$\orbp_6$	&	$\orbp_6$	&	$\orbp_4$	&	$\orbp_5$	\\
		$\orbp_7$	&	$\orbp_7$	&	$\orbp_7$	&	$\orbp_7$	\\ \hline
	\end{tabular}
	\end{center}

	
%	\begin{align}
%		D(E) = \begin{pmatrix}
%			1 & 0 & 0 & 0 & 0 & 0 & 0 \\
%			0 & 1 & 0 & 0 & 0 & 0 & 0 \\
%			0 & 0 & 1 & 0 & 0 & 0 & 0 \\
%			0 & 0 & 0 & 1 & 0 & 0 & 0 \\
%			0 & 0 & 0 & 0 & 1 & 0 & 0 \\
%			0 & 0 & 0 & 0 & 0 & 1 & 0 \\
%			0 & 0 & 0 & 0 & 0 & 0 & 1
%		\end{pmatrix} , \\
%		D(C_3) = \begin{pmatrix}
%			0 & 1 & 0 & 0 & 0 & 0 & 0 \\
%			0 & 0 & 1 & 0 & 0 & 0 & 0 \\
%			1 & 0 & 0 & 0 & 0 & 0 & 0 \\
%			0 & 0 & 0 & 0 & 1 & 0 & 0 \\
%			0 & 0 & 0 & 0 & 0 & 1 & 0 \\
%			0 & 0 & 0 & 1 & 0 & 0 & 0 \\
%			0 & 0 & 0 & 0 & 0 & 0 & 1
%		\end{pmatrix} , \\
%		D(C^2_3) = \begin{pmatrix}
%			0 & 0 & 1 & 0 & 0 & 0 & 0 \\
%			1 & 0 & 0 & 0 & 0 & 0 & 0 \\
%			0 & 1 & 0 & 0 & 0 & 0 & 0 \\
%			0 & 0 & 0 & 0 & 0 & 1 & 0 \\
%			0 & 0 & 0 & 1 & 0 & 0 & 0 \\
%			0 & 0 & 0 & 0 & 1 & 0 & 0 \\
%			0 & 0 & 0 & 0 & 0 & 0 & 1
%		\end{pmatrix} .
%	\end{align}
	
	\begin{center}
	\begin{tabular}{cccc} \hline
		$R$		&	$E$	&	$C_3$		&	$C^2_3$		\\ \hline
		$D(R)$	& $\begin{pmatrix}
			1 & 0 & 0 & 0 & 0 & 0 & 0 \\
			0 & 1 & 0 & 0 & 0 & 0 & 0 \\
			0 & 0 & 1 & 0 & 0 & 0 & 0 \\
			0 & 0 & 0 & 1 & 0 & 0 & 0 \\
			0 & 0 & 0 & 0 & 1 & 0 & 0 \\
			0 & 0 & 0 & 0 & 0 & 1 & 0 \\
			0 & 0 & 0 & 0 & 0 & 0 & 1
		\end{pmatrix}$ & $\begin{pmatrix}
			0 & 1 & 0 & 0 & 0 & 0 & 0 \\
			0 & 0 & 1 & 0 & 0 & 0 & 0 \\
			1 & 0 & 0 & 0 & 0 & 0 & 0 \\
			0 & 0 & 0 & 0 & 1 & 0 & 0 \\
			0 & 0 & 0 & 0 & 0 & 1 & 0 \\
			0 & 0 & 0 & 1 & 0 & 0 & 0 \\
			0 & 0 & 0 & 0 & 0 & 0 & 1
		\end{pmatrix}$ & $\begin{pmatrix}
			0 & 0 & 1 & 0 & 0 & 0 & 0 \\
			1 & 0 & 0 & 0 & 0 & 0 & 0 \\
			0 & 1 & 0 & 0 & 0 & 0 & 0 \\
			0 & 0 & 0 & 0 & 0 & 1 & 0 \\
			0 & 0 & 0 & 1 & 0 & 0 & 0 \\
			0 & 0 & 0 & 0 & 1 & 0 & 0 \\
			0 & 0 & 0 & 0 & 0 & 0 & 1
		\end{pmatrix}$ \\ \hline
	\end{tabular}
	\end{center}
	
	
	\end{solution}	

\end{document}