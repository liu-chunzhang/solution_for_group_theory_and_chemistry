\documentclass[a4paper]{book}

\usepackage{afterpage}
\usepackage[hypcap=false]{caption}
\usepackage{enumitem}	% 定制enumerate标号
\usepackage{geometry}
\geometry{%
	left=2cm,%
	right=2cm,%
	top=2cm,%
	bottom=2cm,%
	bindingoffset=0cm
}
\usepackage{hyperref}
\hypersetup{
    colorlinks=true,            %链接颜色
    linkcolor=blue,             %内部链接
    filecolor=magenta,          %本地文档
    urlcolor=cyan,              %网址链接
    pdftitle={Overleaf Example},
    pdfpagemode=FullScreen,
}
\usepackage[none]{hyphenat}	% 阻止长单词分在两行
\usepackage{longtable}
\usepackage{mathrsfs}	% 提供\mathscr字体
\usepackage[version=4]{mhchem}
\usepackage{multirow}
\usepackage{subcaption}

\RequirePackage[many]{tcolorbox}
\tcbset{
    boxed title style={colback=magenta},
	breakable,
	enhanced,
	sharp corners,
	attach boxed title to top left={yshift=-\tcboxedtitleheight,  yshifttext=-.75\baselineskip},
	boxed title style={boxsep=1pt,sharp corners},
    fonttitle=\bfseries\sffamily,
}

\definecolor{skyblue}{rgb}{0.54, 0.81, 0.94}

\newtcolorbox[auto counter, number within=chapter, number format=\arabic]{exercise}[1][]{
    title={Exercise~\thetcbcounter},
    colframe=skyblue,
    colback=skyblue!12!white,
    boxed title style={colback=skyblue},
    overlay unbroken and first={
        \node[below right,font=\small,color=skyblue,text width=.8\linewidth]
        at (title.north east) {#1};
    }
}

\newtcolorbox[auto counter, number within=chapter, number format=\arabic]{solution}[1][]{
%    top=2ex,
%    boxrule=0pt,
%    leftrule=1.4pt,
    title={Solution~\thetcbcounter},
    colframe=teal!60!green,
    colback=green!12!white,
    boxed title style={colback=teal!60!green},
    overlay unbroken and first={
        \node[below right,font=\small,color=red,text width=.8\linewidth]
        at (title.north east) {#1};
    }
}

\newcommand{\AO}{{\rm AO}}
\newcommand{\Heff}{H^{\rm eff,\pi}}
\newcommand{\Hp}{H^\prime}
\newcommand{\Sp}{S^\prime}
\newcommand{\RRR}{{\rm R}^3}
\newcommand\Figref[1]{Fig \ref{#1}}
\newcommand\Tableref[1]{Table \ref{#1}}
\newcommand{\orb}[1]{{\rm #1}}
\newcommand{\orbs}{\orb{s}}
\newcommand{\orbp}{\orb{p}}
\newcommand{\orbd}{\orb{d}}
\newcommand{\orbf}{\orb{f}}

\allowdisplaybreaks

\begin{document}

	\setcounter{chapter}{12}
	
	% 12.1
	\begin{exercise}
		
	Determine the qualitative form of the molecular orbitals for the square-planar complex $\ce{[Ni(CN)4]^{2-}}$. (Assume that each $\ce{CN}$ ligand provides one $\sigma$-type and two $\pi$-type orbitals to the system.)		
		
	\end{exercise}

	\begin{solution}
	
	\end{solution}
	
	% 12.2
	\begin{exercise}

	Determine the qualitative form of the molecular orbitals for the tetrahedral molecule $\ce{MnO^-4}$. [Assume that each oxygen atom provides just three p-orbitals (set these up so that one points towards the $\ce{Mn}$ and the other two are perpendicular to each other and to the $\ce{Mn}$—$\ce{O}$ axis) and that the $\ce{Mn}$ atom provides 4s and 3d orbitals.] You will be on the right track if you find that
	\begin{align*}
		\Gamma^\sigma 	&= \Gamma^{A_1} \oplus \Gamma^{T_2} , \\
		\Gamma^\pi 		&= \Gamma^{E} \oplus \Gamma^{T_1} \oplus \Gamma^{T_2} .
	\end{align*}
		
	\end{exercise}

	\begin{solution}

	\end{solution}

	% 12.3
	\begin{exercise}
	
	Determine the qualitative form of the molecular orbitals for the eclipsed conformation of ferrocene.
	
	\end{exercise}
	
	\begin{solution}
		
	\end{solution}
	
	% 12.4
	\begin{exercise}
	
	For an octahedral environment the d-orbitals are split into two sets ($\orbd_{e_g}$ and $\orbd_{t_{2g}}$); how would they be split for a square-planar environment?
		
	\end{exercise}
	
	\begin{solution}
	
	\end{solution}
	
	% 12.5
	\begin{exercise}
		
	Set up s qualitative correlation diagram for the $\orbd^3$ configuration in an octahedral environment.		
		
	\end{exercise}
	
	\begin{solution}
	
	\end{solution}


\end{document}